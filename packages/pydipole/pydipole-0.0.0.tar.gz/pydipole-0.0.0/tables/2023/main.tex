% Preamble
\documentclass[journal=jctcce,manuscript=article,layout=traditional]{achemso}

% Make PDFLaTeX builds reproducible
\pdfinfoomitdate=1
\pdfsuppressptexinfo=-1
\pdftrailerid{}
\pdfinfo{/Producer()/Creator()}

% Packages
\usepackage[T1]{fontenc} % Use modern font encodings
\usepackage{mathtools}
\usepackage{graphicx}
\usepackage{lineno}
\usepackage{amsmath}
\usepackage{amssymb}
\usepackage{physics}
\usepackage{mathrsfs}
\usepackage{bm}
\usepackage{threeparttable}
\usepackage{float}
\usepackage{longtable}
\usepackage{multirow}
\usepackage{booktabs}
\usepackage{hyperref}
\usepackage[section]{placeins}
\usepackage{makecell}

\renewcommand{\cellalign}{lc}
\renewcommand{\theadalign}{lc}

% Adjust the page margins
\geometry{
  paperwidth=14in,  % Custom page width
  paperheight=11in, % Custom page height
  left=2cm,
  right=2cm,
  top=2cm,
  bottom=2cm,
}

% Configuration
\SectionNumbersOn

% \warninput command: print standard warning when file does not exist
\makeatletter
% \newcommand{\warninput}[1]{\filename@parse{#1}\InputIfFileExists{#1}{}{\message{LaTeX Warning: File `\filename@base.\ifx\filename@ext\relax tex\else\filename@ext\fi' not found on input line \the\inputlineno}}}
\makeatother


% Settings for hyperref

% Metadata
\title{2023 Table of Dipole Polarizability}

% \author{YingXing Cheng}
% % \affiliation[Ghent University]{Center for Molecular Modeling (CMM), Ghent University, Technologiepark-Zwijnaarde 46, B-9052, Ghent, Belgium}
% \affiliation[University of Stuttgart]{Institute of Applied Analysis and Numerical Simulation Numerical Mathematics for High Performance Computing, University of Stuttgart, Pfaffenwaldring 57, 70569, Stuttgart, Germany}
% % \alsoaffiliation{OPTIONAL}
% % \email{YingXing.Cheng@ugent.be}
% \email{yingxing.cheng@mathematik.uni-stuttgart.de}

% \keywords{TODO}

% Document
\begin{document}

    Static scalar dipole polarizabilities (in atomic units) for neutral atoms.
    If not otherwise indicated by the state symmetry, $M_L (M_J)$-averaged polarizabilities are
    listed; $M_L (M_J)$ respectively denotes that the polarizability for each $M_L (M_J)$ state
    can be found in the reference given. Abbreviations used (uncertainties given here consistently
    as $\pm$ values): exp.: experimentally determined value; NR: nonrelativistic; R: Relativistic,
    DK: Scalar relativistic Douglas-Kroll; MVD: mass-velocity-Darwin; SO: Spin-orbit coupled;
    SF: Dyall's spin-free formalism (scalar relativistic); PP: relativistic pseudopotential;
    LDA: local (spin) density approximation; PW91: Perdew-Wang 91 functional;
    RPA: Random phase approximation; PolPot: Polarization potential; MBPT: many-body
    perturbation theory; CI: configuration interaction; CCSD(T): coupled cluster singles
    doubles (SD) with perturbative triples; FS Fock-space; CEPA: coupled electron pair
    approximation; MR: multi-reference; CAS: complete active space; VPA: variational perturbation
    approach. For all other abbreviations see text or references. If the symmetry of the state
    is not clearly specified as in Doolen's calculations, the calculation was most likely set
    at a specific configuration (orbital occupancy) as listed in the Desclaux tables
    \citenum{Müller1984}, reflecting the ground state symmetry of the specific atom.
    NB: 1 a.u. $= 0.1481847113 \, \text{\AA}^3 = 1.6487773 \times 10^{-41} \,
    \text{C m}^2/\text{V}$.

    \begin{longtable}{lllllrl}
\label{tab:table_2023} \\
\toprule
Z & Atom & Refs. & State & $\alpha$ & Year & Comments \\
\midrule
\endfirsthead
\toprule
Z & Atom & Refs. & State & $\alpha$ & Year & Comments \\
\midrule
\endhead
\midrule
\multicolumn{7}{r}{Continued on next page} \\
\midrule
\endfoot
\bottomrule
\endlastfoot
1 & H & [\citenum{Goldman1989}] & $^2S$ & $4.5$ & 1989 & NR, exact \\
 &  & [\citenum{Goldman1989, Tang2012}] & $^2S_{1/2}$ & $4.49975145989$ & 2012 & \makecell{R, Dirac, variational, Slater/B-splines \\(more digits are given in Ref.~\citenum{Tang2012})} \\
 &  & [\citenum{Tang2012}] & $^2S_{1/2}$ & $4.500170623$ & 2012 & R, Dirac (as above), but with finite mass correction added for the $^1$H isotope \\
 &  & [\citenum{Filippin2014}] & $^2S_{1/2}$ & $4.4997519518$ & 2014 & R, Dirac, Lagrange mesh method (more digits are given in this paper) \\
 &  & [\citenum{Schwerdtfeger2019}] & $^2S_{1/2}$ & $4.50711 \pm 0.00003$ & 2019 & recommended \\
 &  & [\citenum{Li2024}] & $^2S_{1/2}$ & $4.511 \pm 0.004$ & 2024 & NR, neural network QMC, DS (DeepSolid) method \\
2 & He & [\citenum{Newell1965, Schooley1992}] & $^1S_0$ & $1.383746 \pm 0.000007$ & 1992 & exp. \\
 &  & [\citenum{Pachucki2000}] & $^1S_0$ & $1.383191$ & 2000 & R, Dirac, Breit-Pauli, QED, mass pol, correlated basis ($^4$He) \\
 &  & [\citenum{Lach2004b}] & $^1S_0$ & $1.38376079 \pm 0.00000023$ & 2004 & \makecell{R, Dirac, Breit-Pauli, QED, mass pol, \\exponentially correlated Slater functions ($^4$He)} \\
 &  & [\citenum{Schmidt2007}] & $^1S_0$ & $1.383759 \pm 0.000013$ & 2007 & exp. \\
 &  & [\citenum{Puchalski2016}] & $^1S_0$ & $1.3837295330 \pm 0.000000001$ & 2016 & R, Dirac, Breit, QED, recoil, ... ($^4$He) \\
 &  & [\citenum{Gaiser2018}] & $^1S_0$ & $1.3837616 \pm 0.0000027$ & 2018 & exp. \\
 &  & [\citenum{Schwerdtfeger2019}] & $^1S_0$ & $1.38375 \pm 0.00002$ & 2019 & recommended \\
 &  & [\citenum{Puchalski2020}] & $^1S_0$ & $1.38376078 \pm 0.00000014$ & 2020 & R, Dirac, Breit-Pauli, QED + finite nuclear size correction \\
3 & Li & [\citenum{Molof1974}] & $^2S_{1/2}$ & $164.0 \pm 3.4$ & 1974 & exp. \\
 &  & [\citenum{Komasa2001, Lim1999}] & $^2S$ & $164.05$ & 2001 & NR, exponentially correlated Gaussians [\citenum{{Singer1997}}] + R/DK \\
 &  & [\citenum{Miffre2006a}] & $^2S_{1/2}$ & $164.2 \pm 1.1$ & 2006 & exp. \\
 &  & [\citenum{Zhang2007a}] & $^2S_{1/2}$ & $164.21$ & 2007 & Frozen core Hamiltonian, semi-empirical polarisation potential \\
 &  & [\citenum{Johnson2008}] & $^2S_{1/2}$ & $164.084$ & 2008 & R, Dirac, MBPT, Breit, QED, recoil ($^7$Li) \\
 &  & [\citenum{Puchalski2011}] & $^2S_{1/2}$ & $164.1125 \pm 0.0005$ & 2011 & Hyleraas basis, RMW + Darwin + Breit, QED, recoil ($^7$Li) \\
 &  & [\citenum{Schwerdtfeger2019}] & $^2S_{1/2}$ & $164.1125 \pm 0.0005$ & 2019 & recommended \\
 &  & [\citenum{Smialkowski2021}] & $^2S_{1/2}$ & $164.2$ & 2021 & NR, CCSD(T) \\
 &  & [\citenum{Badhan2022}] & $^2S_{1/2}$ & $162.00 \pm 0.24$ & 2022 & R, Dirac-HF, pertubative singles + doubles method, RPA \\
 &  & [\citenum{Li2024}] & $^2S_{1/2}$ & $165.0 \pm 0.1$ & 2024 & NR, neural network QMC, DS (DeepSolid) method \\
4 & Be & [\citenum{Müller1984}] & $^1S_0$ & $37.29$ & 1984 & All-electron SCF plus valence CI \\
 &  & [\citenum{Tunega1997}] & $^1S_0$ & $37.73 \pm 0.05$ & 1997 & CCSD(T) \\
 &  & [\citenum{Patil2000}] & $^1S_0$ & $37.9$ & 2000 & Model potential \\
 &  & [\citenum{Komasa2001}] & $^1S$ & $37.755$ & 2001 & NR, exponentially correlated Gaussians [\citenum{Singer1997}] \\
 &  & [\citenum{Mitroy2003}] & $^1S_0$ & $37.69$ & 2003 & Combination of ab initio and semi-empirical methods \\
 &  & [\citenum{Bendazzoli2004}] & $^1S_0$ & $37.807$ & 2004 & CI, expanded London formula \\
 &  & [\citenum{Porsev2006}] & $^1S_0$ & $37.76 \pm 0.22$ & 2006 & R, Dirac, CI + MBPT2 + experimental data \\
 &  & [\citenum{Komasa2001, Maroulis2006}] & $^1S_0$ & $37.739 \pm 0.030$ & 2006 & R correction of -0.016 applied to value from ref [\citenum{Komasa2001}] \\
 &  & [\citenum{Sahoo2008}] & $^1S_0$ & $37.80 \pm 0.47$ & 2008 & R, Dirac, CCSD \\
 &  & [\citenum{Singh2013}] & $^1S_0$ & $37.86 \pm 0.17$ & 2013 & R, Dirac, MBPT, CCSD \\
 &  & [\citenum{Jiang2015a, Cheng2013}] & $^1S_0$ & $37.73 \pm 0.04$ & 2015 & Combination of theoretical (CICP) and experimental data \\
 &  & [\citenum{A.Manz2019}] & $^1S_0$ & $37.75$ & 2019 & NR, CCSD \\
 &  & [\citenum{Visentin2019}] & $^1S_0$ & $37.69/37.71$ & 2019 & CCSD(T), R X2C-0/NR-0 \\
 &  & [\citenum{Schwerdtfeger2019}] & $--$ & $37.74 \pm 0.03$ & 2019 & recommended \\
 &  & [\citenum{Smialkowski2021}] & $^1S_0$ & $37.7$ & 2021 & NR, CCSD(T) \\
 &  & [\citenum{Dong2021}] & $^1S_0$ & $37.614$ & 2021 & R, MCDHF \\
 &  & [\citenum{Wu2023}] & $^1S_0$ & $37.787$ & 2023 & R, DF, CICP (CP: core polarization) \\
 &  & [\citenum{Li2024}] & $^1S_0$ & $37.58 \pm 0.03$ & 2024 & NR, neural network QMC, DS (DeepSolid) method \\
5 & B & [\citenum{Werner1976}] & $^2P$ & $20.47$ & 1976 & NR, PNO-CEPA, $M_L$ res. \\
 &  & [\citenum{Das1998}] & $^2P$ & $20.43 \pm 0.11$ & 1998 & NR, CCSD(T), $M_L$ res. \\
 &  & [\citenum{Fleig2005}] & $^2P$ & $20.59$ & 2005 & R, SF, MRCI, $M_L$ res. \\
 &  & [\citenum{Fleig2005}] & $^2P_{1/2}/^2P_{3/2}$ & $20.53/20.54$ & 2005 & R, Dirac, MRCI, $M_J$ res. \\
 &  & [\citenum{A.Manz2019}] & $^2P$ & $20.42$ & 2019 & NR, CCSD \\
 &  & [\citenum{Schwerdtfeger2019}] & $--$ & $20.5 \pm 0.1$ & 2019 & recommended \\
 &  & [\citenum{Ehn2021}] & $^2P$ & $20.480$ & 2021 & NR, CCSD(T), CBS(T, Q), CBS(Q, 5) \\
 &  & [\citenum{Wang2021}] & $^2P$ & $20.30 \pm 0.26$ & 2021 & NR, CCSD(T) \\
6 & C & [\citenum{Andersson1992}] & $^3P$ & $11.39$ & 1992 & NR, CASPT2, $M_L$ res. \\
 &  & [\citenum{Das1998}] & $^3P$ & $11.67 \pm 0.07$ & 1998 & NR, CCSD(T), $M_L$ res. \\
 &  & [\citenum{Thierfelder2008}] & $^3P_0$ & $11.26 \pm 0.20$ & 2008 & R, Dirac + Gaunt, CCSD(T) \\
 &  & [\citenum{A.Manz2019}] & $^3P$ & $11.63$ & 2019 & NR, CCSD \\
 &  & [\citenum{Schwerdtfeger2019}] & $--$ & $11.3 \pm 0.2$ & 2019 & recommended \\
 &  & [\citenum{Wang2021}] & $^3P$ & $11.64 \pm 0.15$ & 2021 & NR, CCSD(T) \\
 &  & [\citenum{Ehn2021}] & $^3P$ & $11.683$ & 2021 & NR, CCSD(T), CBS(T, Q) \\
 &  & [\citenum{Ehn2021}] & $^3P$ & $11.324$ & 2021 & NR, CCSD(T), CBS(Q, 5) \\
 &  & [\citenum{CanalNeto2021}] & $^3P$ & $11.542$ & 2021 & R, CCSD(T) \\
 &  & [\citenum{Mori2023}] & $^3P$ & $11.61$ & 2023 & R, CCC (convergent close-coupling method) \\
7 & N & [\citenum{Molof1974, Alpher1959}] & $^4S_{3/2}$ & $7.6 \pm 0.4$ & 1974 & exp. \\
 &  & [\citenum{Werner1976}] & $^4S$ & $7.43$ & 1976 & NR, PNO-CEPA \\
 &  & [\citenum{Stiehler1995}] & $^4S$ & $7.36$ & 1995 & NR, numerical MCSCF, M. res. \\
 &  & [\citenum{Das1998}] & $^4S$ & $7.26 \pm 0.05$ & 1998 & NR, CCSD(T) \\
 &  & [\citenum{Roos2004}] & $^4S$ & $7.41$ & 2004 & R, DK, CASPT2 \\
 &  & [\citenum{Zeiss1977, Buchachenko2010}] & $^4S_{3/2}$ & $7.28$ & 2010 & exp. \\
 &  & [\citenum{A.Manz2019}] & $^4S$ & $7.21$ & 2019 & NR, CCSD \\
 &  & [\citenum{Schwerdtfeger2019}] & $--$ & $7.4 \pm 0.2$ & 2019 & recommended \\
 &  & [\citenum{Visentin2020}] & $^4S$ & $7.2$ & 2020 & AE, CCSD(T) \\
 &  & [\citenum{Wang2021}] & $^4S$ & $7.25 \pm 0.09$ & 2021 & NR, CCSD(T) \\
 &  & [\citenum{Ehn2021}] & $^4S$ & $7.367$ & 2021 & NR, CCSD(T), CBS(T, Q) \\
 &  & [\citenum{Ehn2021}] & $^4S$ & $7.153$ & 2021 & NR, CCSD(T), CBS(Q, 5) \\
 &  & [\citenum{CanalNeto2021}] & $^4S$ & $7.178$ & 2021 & R, CCSD(T) \\
 &  & [\citenum{Li2024}] & $^4S$ & $7.189 \pm 0.008$ & 2024 & NR, neural network QMC, DS (DeepSolid) method \\
8 & O & [\citenum{Alpher1959}] & $^3P_2$ & $5.2 \pm 0.4$ & 1959 & exp. \\
 &  & [\citenum{Andersson1992}] & $^3P$ & $5.4$ & 1992 & NR, CASPT2, $M_L$ res. \\
 &  & [\citenum{Werner1976, Lide2004}] & $^3P$ & $5.41 \pm 0.11$ & 2004 & NR, PNO-CEPA, $M_L$ res. \\
 &  & [\citenum{Maroulis2006, Das1998}] & $^3P$ & $5.24 \pm 0.04$ & 2006 & NR, CCSD(T), $M_L$ res. \\
 &  & [\citenum{A.Manz2019}] & $^3P$ & $5.15$ & 2019 & NR, CCSD \\
 &  & [\citenum{Schwerdtfeger2019}] & $--$ & $5.3 \pm 0.2$ & 2019 & recommended \\
 &  & [\citenum{Wang2021}] & $^3P$ & $5.21 \pm 0.07$ & 2021 & NR, CCSD(T) \\
 &  & [\citenum{Ehn2021}] & $^3P$ & $5.145$ & 2021 & NR, CCSD(T), CBS(T, Q) \\
 &  & [\citenum{Li2024}] & $^3P$ & $5.236 \pm 0.008$ & 2024 & NR, neural network QMC, DS (DeepSolid) method \\
9 & F & [\citenum{Werner1976}] & $^2P$ & $3.76$ & 1976 & NR, PNO-CEPA, $M_L$ res. \\
 &  & [\citenum{Das1998}] & $^2P$ & $3.70 \pm 0.03$ & 1998 & NR, CCSD(T), $M_L$ res. \\
 &  & [\citenum{MEDVEĎ2000a}] & $^2P$ & $3.76 \pm 0.06$ & 2000 & NR, CASPT2, $M_L$ res. \\
 &  & [\citenum{A.Manz2019}] & $^2P$ & $3.62$ & 2019 & NR, CCSD \\
 &  & [\citenum{Schwerdtfeger2019}] & $--$ & $3.74 \pm 0.08$ & 2019 & recommended \\
 &  & [\citenum{Wang2021}] & $^2P$ & $3.68 \pm 0.05$ & 2021 & NR, CCSD(T) \\
 &  & [\citenum{Ehn2021}] & $^2P$ & $3.701$ & 2021 & NR, CCSD(T), CBS(T, Q) \\
 &  & [\citenum{Ehn2021}] & $^2P$ & $3.655$ & 2021 & NR, CCSD(T), CBS(Q, 5) \\
 &  & [\citenum{Li2024}] & $^2P$ & $3.776 \pm 0.007$ & 2024 & NR, neural network QMC, DS (DeepSolid) method \\
10 & Ne & [\citenum{Werner1976}] & $^1S$ & $2.676$ & 1976 & NR, PNNO-CEPA, $M_L$ res. \\
 &  & [\citenum{Huot1991}] & $^1S_0$ & $2.6669 \pm 0.0008$ & 1991 & exp. \\
 &  & [\citenum{Rice1992}] & $^1S$ & $2.68$ & 1992 & NR, CCSD(T) \\
 &  & [\citenum{Dalgarno1997b}] & $^1S_0$ & $2.663$ & 1997 & exp. \\
 &  & [\citenum{Soldán2001a}] & $^1S_0$ & $2.66053 \pm 0.00001$ & 2001 & CCSD(T), ECP \\
 &  & [\citenum{Hald2003}] & $^1S$ & $2.665$ & 2003 & NR, CC3 \\
 &  & [\citenum{Hald2003, Larsen1999, Nakajima2001}] & $^1S$ & $2.666$ & 2003 & R, CC3 + FCI + DK3 correction \\
 &  & [\citenum{Maroulis2006}] & $^1S_0$ & $2.661 \pm 0.005$ & 2006 & R, CCSD(T) \\
 &  & [\citenum{Gaiser2010}] & $^1S_0$ & $2.66110 \pm 0.00003$ & 2010 & exp. \\
 &  & [\citenum{Chattopadhyay2012a, Orcutt1967}] & $^1S_0$ & $2.677 \pm 0.070$ & 2012 & R, Dirac-Coulomb, non-linear PRCC \\
 &  & [\citenum{Gaiser2018}] & $^1S_0$ & $2.6610570 \pm 0.0000064$ & 2018 & exp. \\
 &  & [\citenum{Schwerdtfeger2019}] & $--$ & $2.66110 \pm 0.00003$ & 2019 & recommended \\
 &  & [\citenum{Lesiuk2020}] & $^1S_0$ & $2.66080 \pm 0.00036$ & 2020 & R, QED, AE-CCSD(T), finite nucl. mass \& size corrections \\
 &  & [\citenum{Wang2021}] & $^1S$ & $2.662 \pm 0.034$ & 2021 & NR, CCSD(T) \\
 &  & [\citenum{Hellmann2022}] & $^1S_0$ & $2.661067 \pm 0.000077$ & 2022 & R, QED, AE-CCSDT(Q), finite nucl. mass \& size corrections \\
 &  & [\citenum{Mori2023}] & $^1S_0$ & $2.661 \pm 0.005$ & 2023 & R, CCSD(T) \\
11 & Na & [\citenum{Ekstrom1995}] & $^2S_{1/2}$ & $162.7 \pm 0.5$ & 1995 & exp. \\
 &  & [\citenum{Derevianko1999}] & $^2S_{1/2}$ & $162.6 \pm 0.3$ & 1999 & R, SD all orders + exp. data \\
 &  & [\citenum{Maroulis2006, Thakkar2005}] & $^2S_{1/2}$ & $162.88 \pm 0.60$ & 2006 & R, CCSD(T) \\
 &  & [\citenum{Holmgren2010}] & $^2S_{1/2}$ & $162.7 \pm 0.1/ \pm 1.2$ & 2010 & exp. \\
 &  & [\citenum{Ma2015}] & $^2S_{1/2}$ & $161 \pm 7.5$ & 2015 & exp. \\
 &  & [\citenum{Schwerdtfeger2019}] & $--$ & $162.7 \pm 0.5$ & 2019 & recommended \\
 &  & [\citenum{Smialkowski2021}] & $^2S_{1/2}$ & $163.9$ & 2021 & NR, CCSD(T) \\
 &  & [\citenum{Badhan2022}] & $^2S_{1/2}$ & $162.44 \pm 0.16$ & 2022 & R, Dirac-HF, pertubative singles + doubles method, RPA \\
12 & Mg & [\citenum{Lundin1973}] & $^1S_0$ & $71.5 \pm 3.5$ & 1973 & exp. \\
 &  & [\citenum{Reinsch1976}] & $^1S$ & $71.32$ & 1976 & NR, PNO-CEPA \\
 &  & [\citenum{Maeder1979}] & $^1S_0$ & $70.5$ & 1979 & NR, CI + pseudo-potential \\
 &  & [\citenum{Archibong1991}] & $^1S$ & $71.7$ & 1991 & NR, MBPT4 \\
 &  & [\citenum{Sadlej1991a}] & $^1S$ & $71.8$ & 1991 & NR, MBPT4 \\
 &  & [\citenum{Stwalley1971, Stiehler1995}] & $^1S_0$ & $75.0 \pm 3.5$ & 1995 & exp. \\
 &  & [\citenum{Derevianko1999}] & $^1S_0$ & $74.9 \pm 2.7$ & 1999 & Hybrid-RCI + MBPT sum rule \\
 &  & [\citenum{Patil2000}] & $^1S_0$ & $72.0$ & 2000 & Model potential \\
 &  & [\citenum{Bromley2002}] & $^1S_0$ & $71.4$ & 2002 & CI, oscillator strength correction \\
 &  & [\citenum{Mitroy2003}] & $^1S_0$ & $71.35$ & 2003 & Combination of ab initio and semi-empirical methods \\
 &  & [\citenum{Roos2004a}] & $^1S$ & $70.90$ & 2004 & R, DK, CASPT2 \\
 &  & [\citenum{Porsev2006, Porsev2006a}] & $^1S_0$ & $70.89$ & 2006 & R, Dirac, CI + MBPT2 + experimental data \\
 &  & [\citenum{Maroulis2006}] & $^1S_0$ & $71.22 \pm 0.36$ & 2006 & R, DK, CCSD(T) \\
 &  & [\citenum{Porsev2006}] & $^1S_0$ & $71.33$ & 2006 & R, Dirac, CI + MBPT2 \\
 &  & [\citenum{Porsev2006}] & $^1S_0$ & $71.3 \pm 0.7$ & 2006 & R, Dirac, CI + MBPT2, recommended \\
 &  & [\citenum{Sahoo2008}] & $^1S_0$ & $73.4 \pm 2.3$ & 2008 & R, Dirac, CCSD \\
 &  & [\citenum{Hohm2012}] & $^1S_0$ & $77.6 \pm 7.8$ & 2012 & exp. \\
 &  & [\citenum{Singh2013}] & $^1S_0$ & $72.54 \pm 0.50$ & 2013 & R, Dirac, MBPT, CCSD \\
 &  & [\citenum{Chattopadhyay2014}] & $^1S_0$ & $70.76$ & 2014 & \makecell{R, Dirac + Breit, perturbed relativistic \\coupled-cluster theory (PRCC)} \\
 &  & [\citenum{Ma2015}] & $^1S_0$ & $59 \pm 16$ & 2015 & exp. \\
 &  & [\citenum{Visentin2019}] & $^1S_0$ & $71.15/71.63$ & 2019 & CCSD(T), R X2C-0/NR-0 \\
 &  & [\citenum{Visentin2019}] & $^1S_0$ & $71.02/71.01$ & 2019 & CCSD(T), R X2C-2/NR-2 \\
 &  & [\citenum{Bala2019}] & $^1S_0$ & $73.0$ & 2019 & R, KRCISD/aug-QZ \\
 &  & [\citenum{Schwerdtfeger2019}] & $--$ & $71.2 \pm 0.4$ & 2019 & recommended \\
 &  & [\citenum{Shukla2020}] & $^1S_0$ & $71.9$ & 2020 & R, MCDF \\
 &  & [\citenum{CanalNeto2021}] & $^1S_0$ & $72.121$ & 2021 & R, CCSD(T) \\
 &  & [\citenum{Neto2021}] & $^1S_0$ & $71.643$ & 2021 & R, CCSD(T) \\
 &  & [\citenum{Smialkowski2021}] & $^1S_0$ & $71.5$ & 2021 & NR, CCSD(T) \\
 &  & [\citenum{Sarkisov2022}] & $^1S_0$ & $81 \pm 17$ & 2022 & exp. \\
13 & Al & [\citenum{Stwalley1971}] & $^2P$ & $62.0$ & 1971 & NR, numerical MCSCF, $M_L$ res. \\
 &  & [\citenum{Reinsch1976}] & $^2P$ & $56.27$ & 1976 & NR, PNO-CEPA \\
 &  & [\citenum{Hibbert1980}] & $^2P$ & $59.47$ & 1980 & NR, MRCI \\
 &  & [\citenum{Milani1990}][92] & $^2P$ & $46 \pm 2$ & 1990 & exp. (see also ref [\citenum{{Ma2015}}]) \\
 &  & [\citenum{Chu2004}] & $^2P$ & $61.0$ & 2004 & SIC-DFT \\
 &  & [\citenum{Fuentealba2004}] & $^2P$ & $58.0 \pm 0.4$ & 2004 & CCSD(T) \\
 &  & [\citenum{Lupinetti2005}] & $^2P$ & $57.74$ & 2005 & NR, CCSD(T), $M_L$ res. \\
 &  & [\citenum{Fleig2005}] & $^2P$ & $55.5$ & 2005 & R, SF, MRCI, $M_L$ res. \\
 &  & [\citenum{Fleig2005}] & $^2P_{1/2}/^2P_{3/2}$ & $55.4 \pm 2.2/55.9 \pm 2.2$ & 2005 & R, Dirac, MRCI, $M_J$ res. \\
 &  & [\citenum{Maroulis2006}] & $^2P$ & $57.79 \pm 0.30$ & 2006 & R, DK, CCSD(T) \\
 &  & [\citenum{Buchachenko2010c}] & $^2P_{1/2}/^2P_{3/2}$ & $57.8 \pm 1.0/58.0 \pm 1.0$ & 2010 & SI-SOCI, $M_L$ res. \\
 &  & [\citenum{Hohm2012, Sarkisov2006}] & $^2P$ & $55.3 \pm 5.5$ & 2012 & exp. \\
 &  & [\citenum{Gould2016b}] & $^2P$ & $58.3$ & 2016 & SIC-DFT (RXH) \\
 &  & [\citenum{A.Manz2019}] & $^2P$ & $57.85$ & 2019 & NR, CCSD \\
 &  & [\citenum{Schwerdtfeger2019}] & $--$ & $57.8 \pm 1.0$ & 2019 & recommended \\
 &  & [\citenum{CanalNeto2021}] & $^2P$ & $58.089$ & 2021 & R, CCSD(T) \\
 &  & [\citenum{Neto2021}] & $^2P$ & $48.455$ & 2021 & R, CCSD(T) \\
 &  & [\citenum{Kumar2022}] & $^2P_{1/2}/^2P_{3/2}$ & $58.8 \pm 1.2/64.7 \pm 1.3$ & 2022 & R, Breit+QED, CCSD \\
 &  & [\citenum{Sarkisov2022}] & $^2P$ & $54 \pm 11$ & 2022 & exp. \\
 &  & [\citenum{Chen2022a}] & $^2P$ & $54.9 \pm 5.3$ & 2022 & exp. \\
 &  & [\citenum{Wang2021, Wang2023b}] & $^2P$ & $47.69 \pm 0.82$ & 2023 & NR, CCSD(T) \\
14 & Si & [\citenum{Reinsch1976}] & $^3P$ & $36.32$ & 1976 & NR, PNO-CEPA, $M_L$ res. \\
 &  & [\citenum{Hibbert1980}] & $^3P$ & $36.95$ & 1980 & NR, MRCI \\
 &  & [\citenum{Andersson1992}] & $^3P$ & $36.54$ & 1992 & NR, CASPT2, $M_L$ res. \\
 &  & [\citenum{Stiehler1995}] & $^3P$ & $38.8$ & 1995 & NR, numerical MCSCF, $M_L$ res. \\
 &  & [\citenum{Maroulis2003}] & $^3P$ & $37.4 \pm 0.1$ & 2003 & NR, CCSD(T), $M_L$ res. \\
 &  & [\citenum{Chu2004}] & $^3P$ & $38.9$ & 2004 & SIC-DFT \\
 &  & [\citenum{Lupinetti2005}] & $^3P$ & $37.17 \pm 0.21$ & 2005 & NR, CCSD(T), $M_L$ res. \\
 &  & [\citenum{Thierfelder2008}] & $^3P_0$ & $37.31 \pm 0.70$ & 2008 & R, Dirac + Gaunt, CCSD(T) \\
 &  & [\citenum{Gould2016b}] & $^3P$ & $37.8$ & 2016 & SIC-DFT (RXH) \\
 &  & [\citenum{A.Manz2019}] & $^3P$ & $37.16$ & 2019 & NR, CCSD \\
 &  & [\citenum{Schwerdtfeger2019}] & $--$ & $37.3 \pm 0.7$ & 2019 & recommended \\
 &  & [\citenum{Wang2021}] & $^3P$ & $36.90 \pm 0.63$ & 2021 & NR, CCSD(T) \\
 &  & [\citenum{CanalNeto2021}] & $^3P$ & $36.803$ & 2021 & R, CCSD(T) \\
15 & P & [\citenum{Reinsch1976}] & $^4S$ & $24.7 \pm 0.5$ & 1976 & NR, PNO-CEPA \\
 &  & [\citenum{Andersson1992}] & $^4S$ & $24.6 \pm 0.2$ & 1992 & NR, CASPT2 \\
 &  & [\citenum{Stiehler1995}] & $^4S$ & $25.5$ & 1995 & NR, numerical MCSCF, $M_L$ res. \\
 &  & [\citenum{Roos2004}] & $^4S$ & $24.9$ & 2004 & R, DK, CASPT2 \\
 &  & [\citenum{Chu2004}] & $^4S$ & $26.11$ & 2004 & SIC-DFT \\
 &  & [\citenum{Lupinetti2005}] & $^4S$ & $24.93 \pm 0.15$ & 2005 & NR, CCSD(T) \\
 &  & [\citenum{Buchachenko2010}] & $^4S$ & $25.06$ & 2010 & R, DK, CASPT2 \\
 &  & [\citenum{Gould2016b}] & $^4S$ & $25.3$ & 2016 & SIC-DFT (RXH) \\
 &  & [\citenum{A.Manz2019}] & $^4S$ & $24.88$ & 2019 & NR, CCSD \\
 &  & [\citenum{Schwerdtfeger2019}] & $--$ & $25 \pm 1$ & 2019 & recommended \\
 &  & [\citenum{Visentin2020}] & $^4S$ & $25.0$ & 2020 & AE, CCSD(T) \\
 &  & [\citenum{Wang2021}] & $^4S$ & $24.95 \pm 0.43$ & 2021 & NR, CCSD(T) \\
 &  & [\citenum{CanalNeto2021}] & $^4S$ & $24.980$ & 2021 & R, CCSD(T) \\
16 & S & [\citenum{Reinsch1976}] & $^3P$ & $19.60$ & 1976 & NR, PNO-CEPA, $M_L$ res. \\
 &  & [\citenum{Andersson1992}] & $^3P$ & $19.6$ & 1992 & NR, CASPT2, $M_L$ res. \\
 &  & [\citenum{MEDVEĎ2000a}] & $^3P$ & $19.6$ & 2000 & NR, CASPT2, $M_L$ res. \\
 &  & [\citenum{Chu2004}] & $^3P$ & $19.72$ & 2004 & SIC-DFT \\
 &  & [\citenum{Lupinetti2005}] & $^3P$ & $19.37 \pm 0.12$ & 2005 & NR, CCSD(T), $M_L$ res. \\
 &  & [\citenum{A.Manz2019}] & $^3P$ & $19.22$ & 2019 & NR, CCSD(T) \\
 &  & [\citenum{Schwerdtfeger2019}] & $--$ & $19.4 \pm 0.1$ & 2019 & recommended \\
 &  & [\citenum{Wang2021}] & $^3P$ & $19.38 \pm 0.33$ & 2021 & NR, CCSD(T) \\
17 & Cl & [\citenum{Reinsch1976}] & $^2P$ & $14.71$ & 1976 & NR, PNO-CEPA, $M_L$ res. \\
 &  & [\citenum{Andersson1992}] & $^2P$ & $14.6$ & 1992 & NR, CASPT2, $M_L$ res. \\
 &  & [\citenum{MEDVEĎ2000a}] & $^2P$ & $14.73$ & 2000 & NR, CASPT2, $M_L$ res. \\
 &  & [\citenum{Chu2004}] & $^2P$ & $14.7$ & 2004 & SIC-DFT \\
 &  & [\citenum{Lupinetti2005}] & $^2P$ & $14.57 \pm 0.10$ & 2005 & NR, CCSD(T), $M_L$ res. \\
 &  & [\citenum{Schwerdtfeger2019}] & $--$ & $14.6 \pm 0.1$ & 2019 & recommended \\
 &  & [\citenum{Wang2021}] & $^2P$ & $14.59 \pm 0.25$ & 2021 & NR, CCSD(T) \\
 &  & [\citenum{CanalNeto2021}] & $^2P$ & $14.582$ & 2021 & R, CCSD(T) \\
18 & Ar & [\citenum{Newell1965}] & $^1S_0$ & $11.083 \pm 0.002$ & 1965 & exp. \\
 &  & [\citenum{Orcutt1967}] & $^1S_0$ & $11.081 \pm 0.005$ & 1967 & exp. \\
 &  & [\citenum{Langhoff1969}] & $^1S_0$ & $11.091$ & 1969 & exp. \\
 &  & [\citenum{Reinsch1976}] & $^1S$ & $11.10$ & 1976 & NR, PNO-CEPA \\
 &  & [\citenum{Hohm1990, Johnston1960}] & $^1S_0$ & $11.070 \pm 0.007$ & 1990 & exp. \\
 &  & [\citenum{Dalgarno1997b}] & $^1S_0$ & $11.080$ & 1997 & exp. \\
 &  & [\citenum{Soldán2001a}] & $^1S$ & $11.08401 \pm 0.00004$ & 2001 & NR, CCSD(T) \\
 &  & [\citenum{Nakajima2001, Soldán2001a}] & $^1S$ & $11.10$ & 2001 & R, CCSD(T) + DK3 correction \\
 &  & [\citenum{Roos2004}] & $^1S$ & $11.1$ & 2004 & R, DK, CASPT2 \\
 &  & [\citenum{Maroulis2006}] & $^1S_0$ & $11.078 \pm 0.010$ & 2006 & exp. \\
 &  & [\citenum{Maroulis2006, Hohm2012, Lupinetti2005}] & $^1S$ & $11.085 \pm 0.060$ & 2012 & R, CCSD(T) \\
 &  & [\citenum{Singh2013}] & $^1S$ & $11.089 \pm 0.004$ & 2013 & R, CCSD(T) \\
 &  & [\citenum{Gaiser2018}] & $^1S_0$ & $11.07718 \pm 0.00064$ & 2018 & exp. \\
 &  & [\citenum{Schwerdtfeger2019}] & $--$ & $11.083 \pm 0.007$ & 2019 & recommended \\
 &  & [\citenum{Wang2021}] & $^1S$ & $11.08 \pm 0.19$ & 2021 & NR, CCSD(T) \\
 &  & [\citenum{Lesiuk2023}] & $^1S$ & $11.0775 \pm 0.0019$ & 2023 & \makecell{R, DKH2, CCSD(T) + Breit-Pauli \\+ QED + finite nuclear mass and size} \\
19 & K & [\citenum{Molof1974}] & $^2S_{1/2}$ & $292.9 \pm 6.1$ & 1974 & exp. \\
 &  & [\citenum{Derevianko1999}] & $^2S_{1/2}$ & $289.1$ & 1999 & RLCCSD \\
 &  & [\citenum{Derevianko1999}] & $^2S_{1/2}$ & $290.2 \pm 0.8$ & 1999 & R, SD all orders + exp. data for electronic transitions \\
 &  & [\citenum{Mitroy2003}] & $^2S_{1/2}$ & $290.0$ & 2003 & Combination of ab initio and semi-empirical methods \\
 &  & [\citenum{Lim2005}] & $^2S$ & $291.1 \pm 1.5$ & 2005 & R, DK, CCSD(T), AE \\
 &  & [\citenum{Derevianko2010}] & $^2S_{1/2}$ & $290.2$ & 2010 & Combination of theoretical and experimental data \\
 &  & [\citenum{Holmgren2010}] & $^2S_{1/2}$ & $290.6 \pm 1.4$ & 2010 & exp. (for hyperfine effects see ref [\citenum{{Jiang2013}}]) \\
 &  & [\citenum{Jiang2013}] & $^2S_{1/2}$ & $290.05$ & 2013 & Oscillator-strength sum rule \\
 &  & [\citenum{Gregoire2015, Gregoire2016}] & $^2S_{1/2}$ & $289.7 \pm 0.3$ & 2016 & exp. \\
 &  & [\citenum{Schwerdtfeger2019}] & $--$ & $289.7 \pm 0.3$ & 2019 & recommended \\
 &  & [\citenum{Smialkowski2021}] & $^2S_{1/2}$ & $289.6$ & 2021 & SR, CCSD(T), ECP \\
 &  & [\citenum{Badhan2022}] & $^2S_{1/2}$ & $290.30 \pm 0.23$ & 2022 & R, Dirac-HF, pertubative singles + doubles method, RPA \\
20 & Ca & [\citenum{Maeder1979}] & $^1S$ & $153.7$ & 1979 & NR, CI + pseudo-potential \\
 &  & [\citenum{Sadlej1991}] & $^1S$ & $152.0$ & 1991 & R, MVD, CCSD + T \\
 &  & [\citenum{Archibong1991}] & $^1S$ & $157$ & 1991 & NR, MBPT4 \\
 &  & [\citenum{Porsev2002}] & $^1S_0$ & $160$ & 2002 & R, CI+MBPT \\
 &  & [\citenum{Bromley2002}] & $^1S$ & $158.6$ & 2002 & CI, oscillator strength correction \\
 &  & [\citenum{Mitroy2003}] & $^1S$ & $159.4$ & 2003 & Combination of ab initio and semi-empirical methods \\
 &  & [\citenum{Moszynski2003}] & $^1S_0$ & $155.3/157.7$ & 2003 & CCSD R/NR \\
 &  & [\citenum{Roos2004a}] & $^1S$ & $163$ & 2004 & R, DK, CASPT2 \\
 &  & [\citenum{Lim2004}] & $^1S_0$ & $158.0$ & 2004 & R, DK + SO, CCSD(T) \\
 &  & [\citenum{Lide2004, Schwartz1974}] & $^1S_0$ & $169 \pm 17$ & 2004 & exp. \\
 &  & [\citenum{Porsev2006, Porsev2006a}] & $^1S_0$ & $155.9$ & 2006 & R, Dirac, CI + MBPT2 + experimental data \\
 &  & [\citenum{Porsev2006}] & $^1S_0$ & $157.1 \pm 1.3$ & 2006 & Hybrid-RCI + MBPT sum rule + experimental data \\
 &  & [\citenum{Porsev2006}] & $^1S_0$ & $159.0$ & 2006 & R, Dirac, CI + MBPT \\
 &  & [\citenum{Maroulis2006}] & $^1S_0$ & $157.9 \pm 0.8$ & 2006 & R, DK, CCSD(T) \\
 &  & [\citenum{Sahoo2008}] & $^1S_0$ & $154.58$ & 2008 & R, Dirac, coupled cluster \\
 &  & [\citenum{Sahoo2008}] & $^1S_0$ & $154.6 \pm 5.4$ & 2008 & R, Dirac, CCSD \\
 &  & [\citenum{Mitroy2008}] & $^1S_0$ & $154.7$ & 2008 & ab initio + experimental data \\
 &  & [\citenum{Derevianko2010}] & $^1S_0$ & $157.1$ & 2010 & Combination of theoretical and experimental data \\
 &  & [\citenum{Singh2013}] & $^1S_0$ & $157.03 \pm 0.80$ & 2013 & R, Dirac, MBPT, CCSD \\
 &  & [\citenum{Chattopadhyay2014}] & $^1S_0$ & $160.77$ & 2014 & \makecell{R, Dirac + Breit, perturbed relativistic \\coupled-cluster theory (PRCC)} \\
 &  & [\citenum{Visentin2019}] & $^1S_0$ & $156.10$ & 2019 & CCSD(T), ECP \\
 &  & [\citenum{Visentin2019}] & $^1S_0$ & $157.61/159.81$ & 2019 & CCSD(T), R X2C-10/NR-10 \\
 &  & [\citenum{Bala2019}] & $^1S_0$ & $157.5$ & 2019 & R, KRCISD/aug-QZ \\
 &  & [\citenum{Schwerdtfeger2019}] & $--$ & $160.8 \pm 4.0$ & 2019 & recommended \\
 &  & [\citenum{Shukla2020}] & $^1S_0$ & $158.2$ & 2020 & R, MCDF \\
 &  & [\citenum{Smialkowski2021}] & $^1S_0$ & $156.2$ & 2021 & SR, CCSD(T), ECP \\
 &  & [\citenum{Zhang2023}] & $^1S_0$ & $159.43 \pm 0.97$ & 2023 & R, CI+MBPT \\
21 & Sc & [\citenum{Chandler1987, Glass1983}] & $^2D^{,3d1}$ & $107.1$ & 1987 & NR, small CI, VPA \\
 &  & [\citenum{Glass1983, Glass1987}] & $^2D^{,3d1}$ & $138.8$ & 1987 & NR, small CI, VPA \\
 &  & [\citenum{Pou-Amérigo1995}] & $^2D^{,3d1}$ & $142 \pm 21$ & 1995 & NR, MCPF \\
 &  & [\citenum{Torrens2002}] & $^2D^{,3d1}$ & $114.00$ & 2002 & Interacting-induced-dipoles polarisation model \\
 &  & [\citenum{Lide2004, Doolen1987}] & $^2D_{3/2}^{,3d1}$ & $120 \pm 30$ & 2004 & R, Dirac, LDA \\
 &  & [\citenum{Kłos2005a}] & $^2D_{3/2}^{,3d1}$ & $121 \pm 12$ & 2005 & R, DK, MRCI \\
 &  & [\citenum{Chu2005}] & $^2D^{,3d1}$ & $105.88$ & 2005 & TD-DFT \\
 &  & [\citenum{Xi-Bo2009}] & $^2D^{,3d1}$ & $115.46$ & 2009 & DFT \\
 &  & [\citenum{Ma2015}] & $^2D^{,3d1}$ & $97.2 \pm 9.5$ & 2015 & exp. \\
 &  & [\citenum{Gould2016a}] & $^2D^{,3d1}$ & $123$ & 2016 & TD-DFT (LEXX) \\
 &  & [\citenum{Chu2004, Gould2016b}] & $^2D_{3/2}^{,3d1}$ & $106.0$ & 2016 & SIC-DFT (RXH) \\
 &  & [\citenum{Gould2016b}] & $^2D^{,3d1}$ & $134.6$ & 2016 & TD-DFT (PGG) \\
 &  & [\citenum{A.Manz2019}] & $^2D^{,3d1}$ & $125.84$ & 2019 & NR, CCSD \\
 &  & [\citenum{Szarek2019}] & $^2D_{3/2}^{,3d1}$ & $138.39$ & 2019 & R, CCSD(T)/ANO-RCC \\
 &  & [\citenum{Schwerdtfeger2019}] & $--$ & $97 \pm 10$ & 2019 & recommended \\
22 & Ti & [\citenum{Chandler1987}] & $^3F^{,3d2}$ & $91.8$ & 1987 & NR, small CI, VPA \\
 &  & [\citenum{Chandler1987}] & $^3F^{,3d2}$ & $91.4$ & 1987 & NR, small CI, VPA \\
 &  & [\citenum{Pou-Amérigo1995}] & $^3F^{,3d2}$ & $114 \pm 17$ & 1995 & NR, MCPF \\
 &  & [\citenum{Lide2004, Doolen1987}] & $^3F_2^{,3d2}$ & $99 \pm 25$ & 2004 & R, Dirac, LDA \\
 &  & [\citenum{Chu2004}] & $^3F^{,3d2}$ & $85.7$ & 2004 & SIC-DFT \\
 &  & [\citenum{Kłos2005a}] & $^3F_2^{,3d2}$ & $102 \pm 10$ & 2005 & R, DK, MRCI \\
 &  & [\citenum{Chu2005}] & $^3F^{,3d2}$ & $94.69$ & 2005 & TD-DFT \\
 &  & [\citenum{Ma2015}] & $^3F_2^{,3d2}$ & $63.4 \pm 3.4$ & 2015 & exp. \\
 &  & [\citenum{Gould2016a}] & $^3F,3d^2$ & $102$ & 2016 & TD-DFT (LEXX) \\
 &  & [\citenum{Gould2016b}] & $^3F,3d^2$ & $89.4$ & 2016 & SIC-DFT (RXH) \\
 &  & [\citenum{Gould2016b}] & $^3F,3d^2$ & $111.4$ & 2016 & TD-DFT (PGG) \\
 &  & [\citenum{A.Manz2019}] & $^3F,3d^2$ & $86.92$ & 2019 & NR, CCSD \\
 &  & [\citenum{Szarek2019}] & $^3F,3d^2$ & $104.01$ & 2019 & R, CCSD(T)/ANO-RCC \\
 &  & [\citenum{Schwerdtfeger2019}] & $--$ & $100 \pm 10$ & 2019 & recommended \\
 &  & [\citenum{CanalNeto2021}] & $^3F,3d^2$ & $106.22$ & 2021 & R, CCSD(T) \\
 &  & [\citenum{Neto2021}] & $^3F,3d^2$ & $98.373$ & 2021 & R, CCSD(T) \\
 &  & [\citenum{Eustice2023}] & $^3F_4,3d^2$ & $100.4 \pm 1.8$ & 2023 & R, Dirac-HF, CI + all-order \\
23 & V & [\citenum{Chandler1987}] & $^4F,3d^3$ & $80.6$ & 1987 & NR, small CI, VPA \\
 &  & [\citenum{Chandler1987}] & $^4F,3d^3$ & $84.6$ & 1987 & NR, small CI, VPA \\
 &  & [\citenum{Pou-Amérigo1995}] & $^4F,3d^3$ & $97 \pm 15$ & 1995 & NR, MCPF \\
 &  & [\citenum{Lide2004, Doolen1987}] & $^4F_{3/2},3d^3$ & $84 \pm 21$ & 2004 & R, Dirac, LDA \\
 &  & [\citenum{Chu2004}] & $^4F,3d^3$ & $72.8$ & 2004 & SIC-DFT \\
 &  & [\citenum{Kłos2005a}] & $^4F_{3/2},3d^3$ & $87.3 \pm 8.7$ & 2005 & R, DK, MRCI \\
 &  & [\citenum{Ma2015}] & $^4F_{3/2},3d^3$ & $68.2 \pm 5.4$ & 2015 & exp. \\
 &  & [\citenum{Gould2016a}] & $^4F,3d^3$ & $87.3$ & 2016 & TD-DFT (LEXX) \\
 &  & [\citenum{Gould2016b}] & $^4F,3d^3$ & $78.2$ & 2016 & SIC-DFT (RXH) \\
 &  & [\citenum{Gould2016b}] & $^4F,3d^3$ & $96.2$ & 2016 & TD-DFT (PGG) \\
 &  & [\citenum{A.Manz2019}] & $^4F,3d^3$ & $86.85$ & 2019 & NR, CCSD \\
 &  & [\citenum{Szarek2019}] & $^4F,3d^3$ & $94.30$ & 2019 & R, CCSD(T)/ANO-RCC \\
 &  & [\citenum{Schwerdtfeger2019}] & $--$ & $87 \pm 10$ & 2019 & recommended \\
 &  & [\citenum{CanalNeto2021}] & $^4F,3d^3$ & $90.298$ & 2021 & R, CCSD(T) \\
 &  & [\citenum{Neto2021}] & $^4F,3d^3$ & $86.798$ & 2021 & R, CCSD(T) \\
24 & Cr & [\citenum{Pou-Amérigo1995}] & $^7S,3d^5$ & $95 \pm 15$ & 1995 & NR, MCPF \\
 &  & [\citenum{Lide2004, Doolen1987}] & $^7S_3,3d^5$ & $78 \pm 20$ & 2004 & R, Dirac, LDA \\
 &  & [\citenum{Chu2004}] & $^7S,3d^5$ & $60.7$ & 2004 & SIC-DFT \\
 &  & [\citenum{Roos2005}] & $^7S_3,3d^5$ & $78.4 \pm 7.8$ & 2005 & DK, CASPT2 \\
 &  & [\citenum{Buchachenko2010}] & $^7S_3,3d^5$ & $83.2$ & 2010 & R, CCSD(T) \\
 &  & [\citenum{Ma2015}] & $^7S_3,3d^5$ & $60 \pm 24$ & 2015 & exp. \\
 &  & [\citenum{Gould2016a}] & $^7S_3,3d^5$ & $78.4$ & 2016 & TD-DFT (LEXX) \\
 &  & [\citenum{Gould2016b}] & $^7S_3,3d^5$ & $70.4$ & 2016 & TD-DFT (PGG) \\
 &  & [\citenum{Gould2016b}] & $^7S_3,3d^5$ & $69.8$ & 2016 & SIC-DFT (RXH) \\
 &  & [\citenum{A.Manz2019}] & $^7S,3d^5$ & $87.77$ & 2019 & NR, CCSD \\
 &  & [\citenum{Szarek2019}] & $^7S,3d^5$ & $96.20$ & 2019 & R, CCSD(T)/ANO-RCC \\
 &  & [\citenum{Schwerdtfeger2019}] & $--$ & $83 \pm 12$ & 2019 & recommended \\
25 & Mn & [\citenum{Chandler1987}] & $^6S,3d^5$ & $65.4$ & 1987 & NR, small CI, VPA \\
 &  & [\citenum{Pou-Amérigo1995}] & $^6S,3d^5$ & $76 \pm 11$ & 1995 & NR, MCPF \\
 &  & [\citenum{Lide2004, Doolen1987}] & $^6S_{5/2},3d^5$ & $63 \pm 16$ & 2004 & R, Dirac, LDA \\
 &  & [\citenum{Chu2004}] & $^6S,3d^5$ & $56.8$ & 2004 & SIC-DFT \\
 &  & [\citenum{Roos2005}] & $^6S_{5/2},3d^5$ & $66.8 \pm 6.7$ & 2005 & DK, CASPT2 \\
 &  & [\citenum{Buchachenko2010}] & $^6S_{5/2},3d^5$ & $68.5$ & 2010 & R, CCSD(T) \\
 &  & [\citenum{Gould2016a}] & $^6S,3d^5$ & $66.8$ & 2016 & TD-DFT (LEXX) \\
 &  & [\citenum{Gould2016b}] & $^6S,3d^5$ & $76.3$ & 2016 & TD-DFT (PGG) \\
 &  & [\citenum{Gould2016b}] & $^6S,3d^5$ & $63.1$ & 2016 & SIC-DFT (RXH) \\
 &  & [\citenum{A.Manz2019}] & $^6S_{5/2},3d^5$ & $83.98$ & 2019 & NR, CCSD \\
 &  & [\citenum{Szarek2019}] & $^6S_{5/2},3d^5$ & $73.55$ & 2019 & R, CCSD(T)/ANO-RCC \\
 &  & [\citenum{Schwerdtfeger2019}] & $--$ & $68 \pm 9$ & 2019 & recommended \\
 &  & [\citenum{CanalNeto2021}] & $^6S,3d^5$ & $70.154$ & 2021 & R, CCSD(T) \\
 &  & [\citenum{Neto2021}] & $^6S,3d^5$ & $63.476$ & 2021 & R, CCSD(T) \\
26 & Fe & [\citenum{Chandler1987}] & $^5D,3d^6$ & $58.4$ & 1987 & NR, small CI, VPA \\
 &  & [\citenum{Pou-Amérigo1995}] & $^5D,3d^6$ & $63.93$ & 1995 & NR, MCPF \\
 &  & [\citenum{Lide2004, Doolen1987}] & $^5D_4,3d^6$ & $57 \pm 14$ & 2004 & R, Dirac, LDA \\
 &  & [\citenum{Chu2004}] & $^5D_4,3d^6$ & $54.4$ & 2004 & SIC-DFT \\
 &  & [\citenum{Calaminici2004}] & $^5D,3d^6$ & $62.65$ & 2004 & NR, DFT, GGA(PW86) \\
 &  & [\citenum{Gould2016a}] & $^5D_4,3d^6$ & $60.4$ & 2016 & TD-DFT (LEXX) \\
 &  & [\citenum{Gould2016b}] & $^5D_4,3d^6$ & $67.8$ & 2016 & TD-DFT (PGG) \\
 &  & [\citenum{Gould2016b}] & $^5D_4,3d^6$ & $56.3$ & 2016 & SIC-DFT (RXH) \\
 &  & [\citenum{A.Manz2019}] & $^5D,3d^6$ & $67.96$ & 2019 & NR, CCSD \\
 &  & [\citenum{Szarek2019}] & $^5D,3d^6$ & $63.82$ & 2019 & R, CCSD(T)/ANO-RCC \\
 &  & [\citenum{Schwerdtfeger2019}] & $--$ & $62 \pm 4$ & 2019 & recommended \\
27 & Co & [\citenum{Chandler1987}] & $^4F,3d^7$ & $52.3$ & 1987 & NR, small CI, VPA \\
 &  & [\citenum{Pou-Amérigo1995}] & $^4F,3d^7$ & $57.71$ & 1995 & NR, MCPF \\
 &  & [\citenum{Lide2004, Doolen1987}] & $^4F_{9/2},3d^7$ & $51 \pm 13$ & 2004 & R, Dirac, LDA \\
 &  & [\citenum{Chu2004}] & $^4F_{9/2},3d^7$ & $48.9$ & 2004 & SIC-DFT \\
 &  & [\citenum{Gould2016a}] & $^4F,3d^7$ & $53.9$ & 2016 & TD-DFT (LEXX) \\
 &  & [\citenum{Gould2016b}] & $^4F,3d^7$ & $60.9$ & 2016 & TD-DFT (PGG) \\
 &  & [\citenum{Gould2016b}] & $^4F,3d^7$ & $50.8$ & 2016 & SIC-DFT (RXH) \\
 &  & [\citenum{A.Manz2019}] & $^4F_{9/2},3d^7$ & $62.03$ & 2019 & NR, CCSD \\
 &  & [\citenum{Szarek2019}] & $^4F_{9/2},3d^7$ & $56.66$ & 2019 & R, CCSD(T)/ANO-RCC \\
 &  & [\citenum{Schwerdtfeger2019}] & $--$ & $55 \pm 4$ & 2019 & recommended \\
28 & Ni & [\citenum{Chandler1987}] & $^3F,3d^8$ & $48.3$ & 1987 & NR, small CI, VPA \\
 &  & [\citenum{Pou-Amérigo1995}] & $^3F,3d^8$ & $51.10$ & 1995 & NR, MCPF \\
 &  & [\citenum{Lide2004, Doolen1987}] & $^3F_4,3d^8$ & $46 \pm 11$ & 2004 & R, Dirac, LDA \\
 &  & [\citenum{Chu2004}] & $^3F_4,3d^8$ & $44.5$ & 2004 & SIC-DFT \\
 &  & [\citenum{Kłos2005a}] & $^3F_4,3d^8$ & $47.4 \pm 4.7$ & 2005 & R, DK, MRCI \\
 &  & [\citenum{Gould2016a}] & $^3F,3d^8$ & $48.4$ & 2016 & TD-DFT (LEXX) \\
 &  & [\citenum{Gould2016b}] & $^3F,3d^8$ & $55.3$ & 2016 & TD-DFT (PGG) \\
 &  & [\citenum{Gould2016b}] & $^3F,3d^8$ & $46.2$ & 2016 & SIC-DFT (RXH) \\
 &  & [\citenum{A.Manz2019}] & $^3F_4,3d^8$ & $57.32$ & 2019 & NR, CCSD \\
 &  & [\citenum{Szarek2019}] & $^3F_4,3d^8$ & $56.57$ & 2019 & R, CCSD(T)/ANO-RCC \\
 &  & [\citenum{Schwerdtfeger2019}] & $--$ & $49 \pm 3$ & 2019 & recommended \\
 &  & [\citenum{CanalNeto2021}] & $^3F,3d^8$ & $50.849$ & 2021 & R, CCSD(T) \\
 &  & [\citenum{Neto2021}] & $^3F,3d^8$ & $46.919$ & 2021 & R, CCSD(T) \\
29 & Cu & [\citenum{Schwerdtfeger1994}] & $^2S_{1/2},3d{{10}}$ & $45.0$ & 1994 & R, PP, QCISD(T) \\
 &  & [\citenum{Pou-Amérigo1995}] & $^2S,3d^{{10}}$ & $53.44$ & 1995 & NR, MCPF \\
 &  & [\citenum{Lide2004, Doolen1987}] & $^2S_{1/2},3d^{{10}}$ & $41 \pm 10$ & 2004 & R, Dirac, LDA \\
 &  & [\citenum{Chu2004}] & $^2S_{1/2},3d^{{10}}$ & $39.5$ & 2004 & SIC-DFT \\
 &  & [\citenum{Roos2005}] & $^2S_{1/2},3d{{10}}$ & $40.7 \pm 4.1$ & 2005 & R, DK, CASPT2 \\
 &  & [\citenum{Kłos2005a}] & $^2S_{1/2},3d{{10}}$ & $43.7 \pm 4.4$ & 2005 & R, DK, MRCI \\
 &  & [\citenum{Maroulis2006, Neogrády1997}] & $^2S_{1/2},3d{{10}}$ & $46.50 \pm 0.35$ & 2006 & R, DK, CCSD(T) \\
 &  & [\citenum{Mohr2009}] & $^2S_{1/2},3d^{{10}}$ & $46.98$ & 2009 & R, DK, CCSD(T) \\
 &  & [\citenum{Mitroy2010a, Zhang2008a}] & $^2S_{1/2},3d^{{10}}$ & $41.65$ & 2010 & CICP \\
 &  & [\citenum{Hohm2012, Sarkisov2006}] & $^2S_{1/2},3d^{{10}}$ & $54.7 \pm 5.5$ & 2012 & exp. \\
 &  & [\citenum{Ma2015}] & $^2S_{1/2},3d^{{10}}$ & $58.7 \pm 4.7$ & 2015 & exp. \\
 &  & [\citenum{Dyugaev2016}] & $^2S,3d^{{10}}$ & $51.8$ & 2016 & semi-empirical \\
 &  & [\citenum{Ernst2016}] & $^2S_{1/2},3d^{{10}}$ & $42.6 \pm 4.3$ & 2016 & DFT B3LYP/aug-cc-pVDZ \\
 &  & [\citenum{Gould2016a}] & $^2S_{1/2},3d{{10}}$ & $41.7$ & 2016 & TD-DFT (LEXX) \\
 &  & [\citenum{Gould2016b}] & $^2S_{1/2},3d{{10}}$ & $46.1$ & 2016 & TD-DFT (PGG) \\
 &  & [\citenum{Gould2016b}] & $^2S_{1/2},3d{{10}}$ & $41.2$ & 2016 & SIC-DFT (RXH) \\
 &  & [\citenum{Schwerdtfeger2019}] & $--$ & $46.5 \pm 0.5$ & 2019 & recommended \\
30 & Zn & [\citenum{Kellö1995}] & $^1S,3d^{{10}}$ & $37.6$ & 1995 & R, MVD, CCSD(T) \\
 &  & [\citenum{Goebel1996}] & $^1S,3d^{{10}}$ & $39.2 \pm 0.8$ & 1996 & NR, CCSD(T), MP2 basis correction \\
 &  & [\citenum{Goebel1996}] & $^1S_0,3d^{{10}}$ & $38.8 \pm 0.8$ & 1996 & exp. \\
 &  & [\citenum{Seth1997}] & $^1S,3d^{{10}}$ & $38.01$ & 1997 & R, PP, CCSD(T) \\
 &  & [\citenum{Ellingsen2001}] & $^1S_0,3d^{{10}}$ & $39.12$ & 2001 & R, MRCI, pseudo-potential \\
 &  & [\citenum{Lide2004, Doolen1987}] & $^1S_0,3d^{{10}}$ & $38 \pm 9$ & 2004 & R, Dirac, LDA \\
 &  & [\citenum{Chu2004}] & $^1S_0,3d^{{10}}$ & $37.7$ & 2004 & SIC-DFT \\
 &  & [\citenum{Roos2005}] & $^1S,3d^{{10}}$ & $38.4$ & 2005 & R, DK, CASPT2 \\
 &  & [\citenum{Maroulis2006, Kellö1995}] & $^1S_0,3d^{{10}}$ & $38.35 \pm 0.29$ & 2006 & R, MVD, CCSD(T) \\
 &  & [\citenum{Singh2014}] & $^1S_0,3d^{{10}}$ & $38.666 \pm 0.096$ & 2014 & R, Dirac, CCSDT \\
 &  & [\citenum{Chattopadhyay2015}] & $^1S_0,3d^{{10}}$ & $38.75$ & 2015 & R, PRCC(T) \\
 &  & [\citenum{Chattopadhyay2015, Qiao2012}] & $^1S_0,3d^{{10}}$ & $38.92$ & 2015 & exp.+fitting \\
 &  & [\citenum{Gould2016b}] & $^1S_0,3d^{{10}}$ & $39.2$ & 2016 & SIC-DFT (RXH) \\
 &  & [\citenum{Szarek2019}] & $^1S_0,3d^{{10}}$ & $41.50$ & 2019 & R, CCSD(T)/ANO-RCC \\
 &  & [\citenum{Schwerdtfeger2019}] & $--$ & $38.67 \pm 0.30$ & 2019 & recommended \\
 &  & [\citenum{Zaremba-Kopczyk2021}] & $^1S,3d^{{10}}$ & $37.7$ & 2021 & ECP, CCSD(T) \\
 &  & [\citenum{Chakraborty2022}] & $^1S_0,3d^{{10}}$ & $38.99 \pm 0.31$ & 2022 & R, NCCSD \\
31 & Ga & [\citenum{Stiehler1995}] & $^2P$ & $54.9 \pm 1.0$ & 1995 & NR, PNO-CEPA, $M_L$ res. \\
 &  & [\citenum{Cernusak2003}] & $^2P$ & $52.91 \pm 0.40$ & 2003 & R, DK, CCSD(T) \\
 &  & [\citenum{Fleig2005}] & $^2P$ & $50.7$ & 2005 & R, SF, MRCI, $M_L$ res. \\
 &  & [\citenum{Fleig2005}] & $^2P_{1/2}/^2P_{3/2}$ & $49.9/51.6$ & 2005 & R, Dirac, MRCI, $M_J$ res. \\
 &  & [\citenum{Buchachenko2010c}] & $^2P_{1/2}/^2P_{3/2}$ & $51.3 \pm 2.0/53.0 \pm 2.0$ & 2010 & SI-SOCI, $M_J$ res. \\
 &  & [\citenum{Borschevsky2012}] & $^2P_{1/2}/^2P_{3/2}$ & $51.1 \pm 1.5/53.4 \pm 3.0$ & 2012 & \makecell{R, Dirac, FSCC, $M_J$ res. \\($J=3/2$: $M_J=3/2$: 41.9, \\$M_J=1/2$: 65.0)} \\
 &  & [\citenum{Ma2015}] & $^2P_{1/2}$ & $46.6 \pm 4.0$ & 2015 & exp. \\
 &  & [\citenum{Gould2016a}] & $^2P_{1/2}$ & $52.1$ & 2016 & TD-DFT (LEXX) \\
 &  & [\citenum{Gould2016b}] & $^2P_{1/2}$ & $56.0$ & 2016 & SIC-DFT (RXH) \\
 &  & [\citenum{A.Manz2019}] & $^2P_{1/2}$ & $53.01$ & 2019 & NR, CCSD \\
 &  & [\citenum{Schwerdtfeger2019}] & $--$ & $50 \pm 3$ & 2019 & recommended \\
 &  & [\citenum{CanalNeto2021}] & $^2P_{1/2}$ & $40.899$ & 2021 & R, CCSD(T) \\
32 & Ge & [\citenum{Stiehler1995}] & $^3P$ & $41.0$ & 1995 & NR, PNO-CEPA, $M_L$ res. \\
 &  & [\citenum{Chu2004}] & $^3P$ & $41.6$ & 2004 & SIC-DFT \\
 &  & [\citenum{Maroulis2006}] & $^3P_0$ & $40.80 \pm 0.82$ & 2006 & R, PNO-CEPA \\
 &  & [\citenum{Thierfelder2008}] & $^3P$ & $39.97$ & 2008 & \makecell{R, DK, CCSD(T), $M_L$ res. \\($M_L = 0$: 32.11, $M_L = 1$: 43.90)} \\
 &  & [\citenum{Thierfelder2008}] & $^3P_0$ & $39.43 \pm 0.80$ & 2008 & R, Dirac Gaunt, CCSD(T) \\
 &  & [\citenum{Gould2016b}] & $^3P$ & $41.2$ & 2016 & SIC-DFT (RXH) \\
 &  & [\citenum{A.Manz2019}] & $^3P_0$ & $39.78$ & 2019 & NR, CCSD \\
 &  & [\citenum{Schwerdtfeger2019}] & $--$ & $40 \pm 1$ & 2019 & recommended \\
33 & As & [\citenum{Stiehler1995}] & $^4S$ & $29.1$ & 1995 & NR, PNO-CEPA \\
 &  & [\citenum{Stiehler1995}] & $^4S$ & $30.5$ & 1995 & NR, numerical MCSCF \\
 &  & [\citenum{Roos2004}] & $^4S$ & $29.8 \pm 0.6$ & 2004 & R, DK, CASPT2 \\
 &  & [\citenum{Chu2004}] & $^4S$ & $31.52$ & 2004 & SIC-DFT \\
 &  & [\citenum{Buchachenko2010}] & $^4S$ & $29.92$ & 2010 & R, DK, CCSD(T) \\
 &  & [\citenum{Buchachenko2010}] & $^4S$ & $29.81$ & 2010 & ECP, CCSD(T) \\
 &  & [\citenum{Gould2016a}] & $^4S$ & $29.6$ & 2016 & TD-DFT (LEXX) \\
 &  & [\citenum{Gould2016b}] & $^4S$ & $30.7$ & 2016 & SIC-DFT (RXH) \\
 &  & [\citenum{A.Manz2019}] & $^4S$ & $29.65$ & 2019 & NR, CCSD \\
 &  & [\citenum{Schwerdtfeger2019}] & $--$ & $30 \pm 1$ & 2019 & recommended \\
 &  & [\citenum{Visentin2020}] & $^4S$ & $29.6$ & 2020 & ECP, CCSD(T) \\
34 & Se & [\citenum{Alpher1959}] & $^3P$ & $26.24 \pm 0.52$ & 1959 & R, MVD, CASPT2, $M_L$ res. \\
 &  & [\citenum{Cuthbertson1997}] & $^3P_2$ & $28.9 \pm 1.0$ & 1997 & exp. \\
 &  & [\citenum{Chu2004}] & $^3P$ & $26.65$ & 2004 & SIC-DFT \\
 &  & [\citenum{Gould2016b}] & $^3P$ & $29.3$ & 2016 & TD-DFT (PGG) \\
 &  & [\citenum{Gould2016b}] & $^3P$ & $24.0$ & 2016 & SIC-DFT (RXH) \\
 &  & [\citenum{A.Manz2019}] & $^3P$ & $25.03$ & 2019 & NR, CCSD \\
 &  & [\citenum{Schwerdtfeger2019}] & $--$ & $28.9 \pm 1.0$ & 2019 & recommended \\
35 & Br & [\citenum{MEDVEĎ2000a}] & $^2P$ & $21.03$ & 2000 & R, MVD, CASPT2, $M_L$ res. \\
 &  & [\citenum{Fleig2002}] & $^2P_{1/2}$ & $21.9$ & 2002 & R, DK, SO-CI \\
 &  & [\citenum{Fleig2002}] & $^2P_{3/2}$ & $21.7$ & 2002 & R, DK, SO-CI, $M_J$ res. \\
 &  & [\citenum{Chu2004}] & $^2P$ & $21.5$ & 2004 & SIC-DFT \\
 &  & [\citenum{Maroulis2006, MEDVEĎ2000a}] & $^2P$ & $21.13 \pm 0.42$ & 2006 & R, MVD, CASPT2 \\
 &  & [\citenum{Gould2016b}] & $^2P$ & $21.6$ & 2016 & TD-DFT (LEXX) \\
 &  & [\citenum{A.Manz2019}] & $^2P$ & $20.4$ & 2019 & NR, CCSD \\
 &  & [\citenum{Schwerdtfeger2019}] & $--$ & $21 \pm 1$ & 2019 & recommended \\
36 & Kr & [\citenum{Orcutt1967}] & $^1S_0$ & $16.766 \pm 0.008$ & 1967 & exp. \\
 &  & [\citenum{Hohm1990}] & $^1S$ & $16.80 \pm 0.13$ & 1990 & R, DK3, CCSD(T) \\
 &  & [\citenum{Huot1991}] & $^1S_0$ & $16.782 \pm 0.005$ & 1991 & exp. \\
 &  & [\citenum{Thakkar1992}] & $^1S_0$ & $16.79$ & 1992 & DOSD (constrained dipole oscillator strength distribution) \\
 &  & [\citenum{Dalgarno1997b, Hohm1990}] & $^1S_0$ & $16.740$ & 1997 & exp. \\
 &  & [\citenum{Dalgarno1997b}] & $^1S_0$ & $16.734$ & 1997 & exp. \\
 &  & [\citenum{Roos2004}] & $^1S$ & $16.6$ & 2004 & R, DK, CASPT2 \\
 &  & [\citenum{Mani2009}] & $^1S_0$ & $16.012$ & 2009 & R, Dirac, CCSD/T \\
 &  & [\citenum{Chattopadhyay2012}] & $^1S_0$ & $16.736$ & 2012 & R, DK3, CCSD(T) \\
 &  & [\citenum{Dzuba2016b}] & $^1S_0$ & $16.47$ & 2016 & R, RPA, PolPot \\
 &  & [\citenum{Schwerdtfeger2019}] & $--$ & $16.78 \pm 0.02$ & 2019 & recommended \\
 &  & [\citenum{Dutta2020}] & $^1S_0$ & $16.800$ & 2020 & R, DHF, MBPT \\
37 & Rb & [\citenum{Molof1974}] & $^2S_{1/2}$ & $319 \pm 6$ & 1974 & exp. \\
 &  & [\citenum{Mitroy2003}] & $^2S$ & $315.7$ & 2003 & Combination of ab initio and semi-empirical methods \\
 &  & [\citenum{Lim2005}] & $^2S$ & $316.2 \pm 3.2$ & 2005 & R, DK, CCSD(T), AE \\
 &  & [\citenum{Maroulis2006}] & $^2S_{1/2}$ & $319.2 \pm 6.1$ & 2006 & exp. \\
 &  & [\citenum{Derevianko1999, Derevianko2010}] & $^2S_{1/2}$ & $318.6 \pm 0.6$ & 2010 & R, SD all orders + exp. data \\
 &  & [\citenum{Holmgren2010}] & $^2S_{1/2}$ & $318.8 \pm 1.4$ & 2010 & exp. \\
 &  & [\citenum{Jiang2015a}] & $^2S$ & $317.0$ & 2015 & Oscillator-strength sum rule \\
 &  & [\citenum{Gregoire2015, Gregoire2016}] & $^2S_{1/2}$ & $319.8 \pm 0.3$ & 2016 & exp. \\
 &  & [\citenum{Schwerdtfeger2019}] & $--$ & $319.8 \pm 0.3$ & 2019 & recommended \\
 &  & [\citenum{Smialkowski2021}] & $^2S_{1/2}$ & $317.4$ & 2021 & SR, CCSD(T), ECP \\
 &  & [\citenum{Kaur2022}] & $^2S_{1/2}$ & $318.5 \pm 0.6$ & 2022 & R, DHF, all orders \\
 &  & [\citenum{Badhan2022}] & $^2S_{1/2}$ & $318.38 \pm 0.38$ & 2022 & R, Dirac-HF, pertubative singles + doubles method, RPA \\
 &  & [\citenum{Hamilton2023}] & $^2S_{1/2}$ & $319.5 \pm 1.5$ & 2023 & \makecell{R, TDHF + Breit + QED + scaling \\+ structure radiation + normaliz.} \\
38 & Sr & [\citenum{Patil2000}] & $^1S_0$ & $193.2$ & 2000 & Model potential \\
 &  & [\citenum{Bromley2002}] & $^1S_0$ & $198.5 \pm 1.3$ & 2002 & CI, oscillator strength correction \\
 &  & [\citenum{Mitroy2003}] & $^1S_0$ & $201.2$ & 2003 & Combination of ab initio and semi-empirical methods \\
 &  & [\citenum{Lim2004}] & $^1S_0$ & $199.4$ & 2004 & R, DK + SO, CCSD(T) \\
 &  & [\citenum{Lim2004}] & $^1S_0$ & $198.85$ & 2004 & R, DK, CCSD(T) \\
 &  & [\citenum{Lide2004}] & $^1S_0$ & $186 \pm 15$ & 2004 & exp. \\
 &  & [\citenum{Maroulis2006, Porsev2002}] & $^1S$ & $199.0 \pm 2.0$ & 2006 & R, CI, MBPT2 \\
 &  & [\citenum{Porsev2006}] & $^1S_0$ & $202.0$ & 2006 & Hybrid-RCI + MBPT sum rule \\
 &  & [\citenum{Sahoo2008}] & $^1S_0$ & $199.71$ & 2008 & R, Dirac, coupled cluster \\
 &  & [\citenum{Porsev2006a, Porsev2008}] & $^1S_0$ & $197.2 \pm 3.6$ & 2008 & R, Dirac, CI + MBPT + experimental data \\
 &  & [\citenum{Mitroy2008}] & $^1S_0$ & $201.6$ & 2008 & Combination of ab initio and experimental results \\
 &  & [\citenum{Mitroy2010}] & $^1S_0$ & $197.6$ & 2010 & CI + core polarisation (corrected to exp. term energies) \\
 &  & [\citenum{Porsev2006, Derevianko2010}] & $^1S_0$ & $197.2 \pm 0.2$ & 2010 & Hybrid-RCI + MBPT sum rule \\
 &  & [\citenum{Singh2013}] & $^1S_0$ & $186.98 \pm 0.85$ & 2013 & R, Dirac, MBPT, CCSD \\
 &  & [\citenum{Cheng2013}] & $^1S_0$ & $197.8$ & 2013 & Combination of theoretical (CICP) and experimental methods \\
 &  & [\citenum{Safronova2013b}] & $^1S_0$ & $197.14 \pm 0.2$ & 2013 & CI + MBPT and experimental results \\
 &  & [\citenum{Safronova2013b}] & $^1S_0$ & $198.9 \pm 2.0$ & 2013 & CI + MBPT-SD and experimental results \\
 &  & [\citenum{Chattopadhyay2014}] & $^1S_0$ & $190.82$ & 2014 & \makecell{R, Dirac + Breit, perturbed relativistic \\coupled-cluster theory (PRCC)} \\
 &  & [\citenum{Jiang2015a}] & $^1S_0$ & $197.9$ & 2015 & Oscillator-strength sum rule \\
 &  & [\citenum{Visentin2019}] & $^1S_0$ & $198.62/198.93$ & 2019 & CCSD(T), ECP/R X2C-28 \\
 &  & [\citenum{Bala2019}] & $^1S_0$ & $196.5$ & 2019 & R, KRCISD/aug-QZ \\
 &  & [\citenum{Schwerdtfeger2019}] & $--$ & $197.2 \pm 0.2$ & 2019 & recommended \\
 &  & [\citenum{Shukla2020}] & $^1S_0$ & $214.5$ & 2020 & R, MCDF \\
 &  & [\citenum{CanalNeto2021}] & $^1S_0$ & $203.16$ & 2021 & R, CCSD(T) \\
39 & Y & [\citenum{Lide2004, Doolen1987}] & $^2D_{3/2},4d^1$ & $153 \pm 38$ & 2004 & R, Dirac, LDA \\
 &  & [\citenum{Li2009}] & $^2D_{3/2},4d^1$ & $140.94$ & 2009 & DFT, ECP \\
 &  & [\citenum{Hohm2012, Chu2007}] & $^2D_{3/2},4d^1$ & $139 \pm 28$ & 2012 & TD-DFT \\
 &  & [\citenum{Ma2015}] & $^2D_{3/2},4d^1$ & $163 \pm 12$ & 2015 & exp. \\
 &  & [\citenum{Gould2016b}] & $^2D_{3/2},4d^1$ & $134.9$ & 2016 & SIC-DFT (RXH) \\
 &  & [\citenum{Gould2016a}] & $^2D_{3/2},4d^1$ & $163$ & 2016 & TD-DFT (LEXX) \\
 &  & [\citenum{Gould2016b}] & $^2D_{3/2},4d^1$ & $134.9$ & 2016 & SIC-DFT (RXH) \\
 &  & [\citenum{Gould2016b}] & $^2D_{3/2},4d^1$ & $126.74$ & 2016 & TD-DFT (PGG) \\
 &  & [\citenum{gobre2016efficient}] & $^2D_{3/2},4d^1$ & $163 \pm 12$ & 2016 & LR-CCSD \\
 &  & [\citenum{Schwerdtfeger2019}] & $--$ & $162 \pm 12$ & 2019 & recommended \\
40 & Zr & [\citenum{Lide2004, Doolen1987}] & $^3F_2, 4d^2$ & $121 \pm 30$ & 2004 & R, Dirac, LDA \\
 &  & [\citenum{Ma2015}] & $^3F_2, 4d^2$ & $112 \pm 13$ & 2015 & exp. \\
 &  & [\citenum{gobre2016efficient}] & $^3F_2, 4d^2$ & $119.97$ & 2016 & LR-CCSD \\
 &  & [\citenum{Gould2016a}] & $^3F_2, 4d^2$ & $112$ & 2016 & TD-DFT (LEXX) \\
 &  & [\citenum{Gould2016b}] & $^3F_2, 4d^2$ & $109.8$ & 2016 & SIC-DFT (RXH) \\
 &  & [\citenum{Gould2016b}] & $^3F_2, 4d^2$ & $130.5$ & 2016 & TD-DFT (PGG) \\
 &  & [\citenum{Schwerdtfeger2019}] & $--$ & $112 \pm 13$ & 2019 & recommended \\
41 & Nb & [\citenum{Lide2004, Doolen1987}] & $^6D_{1/2}, 4d^4$ & $106 \pm 27$ & 2004 & R, Dirac, LDA \\
 &  & [\citenum{Ma2015}] & $^6D_{1/2}, 4d^4$ & $97.9 \pm 7.4$ & 2015 & exp. \\
 &  & [\citenum{gobre2016efficient}] & $^6D_{1/2}, 4d^4$ & $101.60$ & 2016 & LR-CCSD \\
 &  & [\citenum{Gould2016a}] & $^6D_{1/2}, 4d^4$ & $97.9$ & 2016 & TD-DFT (LEXX) \\
 &  & [\citenum{Gould2016b}] & $^6D_{1/2}, 4d^4$ & $99.6$ & 2016 & TD-DFT (PGG) \\
 &  & [\citenum{Gould2016b}] & $^6D_{1/2}, 4d^4$ & $95.5$ & 2016 & SIC-DFT (RXH) \\
 &  & [\citenum{A.Manz2019}] & $^6D_{1/2}, 4d^4$ & $106.43$ & 2019 & ECP, CCSD \\
 &  & [\citenum{Schwerdtfeger2019}] & $--$ & $98 \pm 8$ & 2019 & recommended \\
42 & Mo & [\citenum{Liepack1956}] & $^7S_3, 4d^5$ & $61 \pm 10$ & 1956 & exp. \\
 &  & [\citenum{Lide2004, Doolen1987}] & $^7S_3, 4d^5$ & $86 \pm 22$ & 2004 & R, Dirac, LDA \\
 &  & [\citenum{Buchachenko2010}] & $^7S_3, 4d^5$ & $84$ & 2010 & R, CCSD(T) \\
 &  & [\citenum{Buchachenko2010}] & $^7S_3, 4d^5$ & $79$ & 2010 & MRCI \\
 &  & [\citenum{Hohm2012, Roos2005}] & $^7S, 4d^5$ & $73 \pm 11$ & 2012 & R, DK, CASPT2 \\
 &  & [\citenum{Ma2015}] & $^7S_3, 4d^5$ & $87.1 \pm 6.1$ & 2015 & exp. \\
 &  & [\citenum{gobre2016efficient}] & $^7S_3, 4d^5$ & $88.42$ & 2016 & LR-CCSD \\
 &  & [\citenum{Gould2016a}] & $^7S_3, 4d^5$ & $87.1$ & 2016 & TD-DFT (LEXX) \\
 &  & [\citenum{Gould2016b}] & $^7S_3, 4d^5$ & $82.7$ & 2016 & TD-DFT (PGG) \\
 &  & [\citenum{Gould2016b}] & $^7S_3, 4d^5$ & $79.0$ & 2016 & SIC-DFT (RXH) \\
 &  & [\citenum{A.Manz2019}] & $^7S_3, 4d^5$ & $85.93$ & 2019 & ECP, CCSD \\
 &  & [\citenum{Schwerdtfeger2019}] & $--$ & $87 \pm 6$ & 2019 & recommended \\
 &  & [\citenum{Neto2021}] & $^7S_3, 4d^5$ & $84.355$ & 2021 & R, CCSD(T) \\
 &  & [\citenum{Sarkisov2022}] & $^7S_3, 4d^5$ & $76 \pm 15$ & 2022 & exp. \\
43 & Tc & [\citenum{Lide2004, Doolen1987}] & $^6S_{5/2}, 4d^5$ & $77 \pm 20$ & 2004 & R, Dirac, LDA \\
 &  & [\citenum{Buchachenko2010}] & $^6S_{5/2}, 4d^5$ & $78.6$ & 2010 & R, CCSD(T) \\
 &  & [\citenum{Hohm2012, Roos2005}] & $^6S,4d^5$ & $80 \pm 12$ & 2012 & R, DK, CASPT2 \\
 &  & [\citenum{Gould2016a}] & $^6S_{5/2}, 4d^5$ & $79.6$ & 2016 & TD-DFT (LEXX) \\
 &  & [\citenum{Gould2016b}] & $^6S_{5/2}, 4d^5$ & $93.9$ & 2016 & TD-DFT (PGG) \\
 &  & [\citenum{Gould2016b}] & $^6S_{5/2}, 4d^5$ & $78.5$ & 2016 & SIC-DFT (RXH) \\
 &  & [\citenum{gobre2016efficient}] & $^6S_{5/2}, 4d^5$ & $80.08$ & 2016 & LR-CCSD \\
 &  & [\citenum{A.Manz2019}] & $^6S_{5/2}, 4d^5$ & $80.9$ & 2019 & ECP, CCSD \\
 &  & [\citenum{Schwerdtfeger2019}] & $--$ & $79 \pm 10$ & 2019 & recommended \\
 &  & [\citenum{CanalNeto2021}] & $^6S_{5/2}, 4d^5$ & $71.113$ & 2021 & R, CCSD(T) \\
 &  & [\citenum{Neto2021}] & $^6S_{5/2}, 4d^5$ & $65.158$ & 2021 & R, CCSD(T) \\
44 & Ru & [\citenum{Lide2004, Doolen1987}] & $^5F_5, 4d^7$ & $65 \pm 16$ & 2004 & R, Dirac, LDA \\
 &  & [\citenum{Gould2016a}] & $^5F_5, 4d^7$ & $72.3$ & 2016 & TD-DFT (LEXX) \\
 &  & [\citenum{Gould2016b}] & $^5F_5, 4d^7$ & $69.5$ & 2016 & TD-DFT (PGG) \\
 &  & [\citenum{Gould2016b}] & $^5F_5, 4d^7$ & $71.4$ & 2016 & SIC-DFT (RXH) \\
 &  & [\citenum{gobre2016efficient}] & $^5F_5, 4d^7$ & $65.89$ & 2016 & LR-CCSD \\
 &  & [\citenum{A.Manz2019}] & $^5F_5, 4d^7$ & $71.27$ & 2019 & ECP, CCSD \\
 &  & [\citenum{Schwerdtfeger2019}] & $--$ & $72 \pm 10$ & 2019 & recommended \\
45 & Rh & [\citenum{Lide2004, Doolen1987}] & $^4F_{9/2}, 4d^8$ & $58 \pm 15$ & 2004 & R, Dirac, LDA \\
 &  & [\citenum{Ma2015}] & $^4F_{9/2}, 4d^8$ & $11 \pm 22$ & 2015 & exp. (an unusually low value was obtained) \\
 &  & [\citenum{Gould2016a}] & $^4F_{9/2}, 4d^8$ & $66.4$ & 2016 & TD-DFT (LEXX) \\
 &  & [\citenum{Gould2016b}] & $^4F_{9/2}, 4d^8$ & $66.2$ & 2016 & TD-DFT (PGG) \\
 &  & [\citenum{Gould2016b}] & $^4F_{9/2}, 4d^8$ & $65.7$ & 2016 & SIC-DFT (RXH) \\
 &  & [\citenum{gobre2016efficient}] & $^4F_{9/2}, 4d^8$ & $56.10$ & 2016 & LR-CCSD \\
 &  & [\citenum{A.Manz2019}] & $^4F_{9/2}, 4d^8$ & $61.94$ & 2019 & ECP, CCSD \\
 &  & [\citenum{Schwerdtfeger2019}] & $--$ & $66 \pm 10$ & 2019 & recommended \\
46 & Pd & [\citenum{Lide2004, Doolen1987}] & $^1S_0, 4d^{10}$ & $32 \pm 8$ & 2004 & R, Dirac, LDA \\
 &  & [\citenum{Bast2008}] & $^1S_0, 4d^{10}$ & $26.612$ & 2008 & NR, ECP, CCSD(T) \\
 &  & [\citenum{Granatier2011}] & $^1S_0, 4d^{10}$ & $24.581$ & 2011 & R, DK \\
 &  & [\citenum{Gould2016a}] & $^1S_0, 4d^{10}$ & $61.7$ & 2016 & TD-DFT (LEXX) \\
 &  & [\citenum{Gould2016b}] & $^1S_0, 4d^{10}$ & $20.0$ & 2016 & TD-DFT (PGG) \\
 &  & [\citenum{Gould2016b}] & $^1S_0, 4d^{10}$ & $61.1$ & 2016 & SIC-DFT (RXH) \\
 &  & [\citenum{gobre2016efficient}] & $^1S_0, 4d^{10}$ & $23.68$ & 2016 & LR-CCSD \\
 &  & [\citenum{Jerabek2018b}] & $^1S_0, 4d^{10}$ & $26.14 \pm 0.10$ & 2018 & CCSDTQDP, DKH2 + Gaunt, CBS \\
 &  & [\citenum{A.Manz2019}] & $^1S_0, 4d^{10}$ & $24.36$ & 2019 & ECP, CCSD \\
 &  & [\citenum{Schwerdtfeger2019}] & $--$ & $26.14 \pm 0.10$ & 2019 & recommended \\
 &  & [\citenum{Sarkisov2022}] & $^1S_0, 4d^{10}$ & $43 \pm 9$ & 2022 & exp. \\
47 & Ag & [\citenum{Neogrády1997}] & $^2S_{1/2}, 4d^{10}$ & $55.3 \pm 0.5$ & 1997 & R, DK, CCSD(T) \\
 &  & [\citenum{Roos2005}] & $^2S, 4d^{10}$ & $36.7$ & 2005 & R, DK, CCSD(T) \\
 &  & [\citenum{Maroulis2006, Neogrády1997}] & $^2S, 4d^{10}$ & $52.46 \pm 0.52$ & 2006 & R, DK, CCSD(T) \\
 &  & [\citenum{Zhang2008a}] & $^2S_{1/2}, 4d^{10}$ & $46.17$ & 2008 & CICP \\
 &  & [\citenum{Schwerdtfeger1994, Mohr2009}] & $^2S, 4d^{10}$ & $52.2$ & 2009 & R, PP, QCISD(T) \\
 &  & [\citenum{Bezchastnov2010}] & $^2S_{1/2}, 4d^{10}$ & $56 \pm 14$ & 2010 & exp. \\
 &  & [\citenum{Hohm2012}] & $^2S_{1/2}, 4d^{10}$ & $63.1 \pm 6.3$ & 2012 & exp. \\
 &  & [\citenum{Ma2015}] & $^2S_{1/2}, 4d^{10}$ & $45.9 \pm 7.4$ & 2015 & exp. \\
 &  & [\citenum{Dyugaev2016}] & $^2S, 4d^{10}$ & $55.2$ & 2016 & Semi-empirical \\
 &  & [\citenum{Gould2016a}] & $^2S_{1/2}, 4d^{10}$ & $46.2$ & 2016 & TD-DFT (LEXX) \\
 &  & [\citenum{Gould2016b}] & $^2S_{1/2}, 4d^{10}$ & $63.3$ & 2016 & TD-DFT (PGG) \\
 &  & [\citenum{Gould2016b}] & $^2S_{1/2}, 4d^{10}$ & $57.3$ & 2016 & SIC-DFT (RXH) \\
 &  & [\citenum{gobre2016efficient}] & $^2S_{1/2}, 4d^{10}$ & $50.60$ & 2016 & LR-CCSD \\
 &  & [\citenum{A.Manz2019}] & $^2S_{1/2}, 4d^{10}$ & $55$ & 2019 & ECP, CCSD \\
 &  & [\citenum{Schwerdtfeger2019}] & $--$ & $55 \pm 8$ & 2019 & recommended \\
 &  & [\citenum{Tomza2021, Smialkowski2021}] & $^2S_{1/2}, 4d^{10}$ & $50.2$ & 2021 & SR, ECP, CCSD(T) \\
 &  & [\citenum{Dzuba2021}] & $^2S_{1/2}, 4d^{10}$ & $50.6$ & 2021 & R, CI+MBPT \\
 &  & [\citenum{Lide2004, Bromley2002b}][2023114] & $^2S_{1/2}, 4d^{10}$ & $48.4$ & 2023 & R, Dirac, LDA \\
48 & Cd & [\citenum{Kellö1995}] & $^1S, 4d^{10}$ & $46.8$ & 1995 & R, MVD, CCSD(T) \\
 &  & [\citenum{Goebel1995}] & $^1S_0, 4d^{10}$ & $49.7 \pm 1.6$ & 1995 & exp. \\
 &  & [\citenum{Goebel1995a}] & $^1S_0, 4d^{10}$ & $48.2 \pm 1.1$ & 1995 & exp. \\
 &  & [\citenum{Seth1997}] & $^1S, 4d^{10}$ & $46.25$ & 1997 & R, PP, CCSD(T) \\
 &  & [\citenum{Bromley2002a, Goebel1995a}] & $^1S_0, 4d^{10}$ & $45.3 \pm 1.4$ & 2002 & exp. \\
 &  & [\citenum{Moszynski2003}] & $^1S_0, 4d^{10}$ & $45.91/53.99$ & 2003 & CCSD R/NR \\
 &  & [\citenum{Roos2005}] & $^1S, 4d^{10}$ & $46.9$ & 2005 & R, DK, CASPT2 \\
 &  & [\citenum{Maroulis2006, Kellö1995}] & $^1S_0, 4d^{10}$ & $47.55 \pm 0.48$ & 2006 & R, MVD, CCSD(T) \\
 &  & [\citenum{Ye2008}] & $^1S_0, 4d^{10}$ & $44.63$ & 2008 & R, DHF, CPMP \\
 &  & [\citenum{Singh2014}] & $^1S_0, 4d^{10}$ & $45.86 \pm 0.15$ & 2014 & R, DF, CCSD(T), MBPT3 \\
 &  & [\citenum{Gould2016a}] & $^1S_0, 4d^{10}$ & $46.7$ & 2016 & TD-DFT (LEXX) \\
 &  & [\citenum{Sahoo2018b}] & $^1S_0, 4d^{10}$ & $46.02 \pm 0.50$ & 2018 & R, DHF, CCSD(T) \\
 &  & [\citenum{A.Manz2019}] & $^1S_0, 4d^{10}$ & $48.3$ & 2019 & ECP, CCSD \\
 &  & [\citenum{Schwerdtfeger2019}] & $--$ & $46 \pm 2$ & 2019 & recommended \\
 &  & [\citenum{Dutta2020}] & $^1S_0, 4d^{10}$ & $39.79$ & 2020 & LR-CCSD \\
 &  & [\citenum{Guo2021}] & $^1S_0, 4d^{10}$ & $45.92 \pm 0.10$ & 2021 & R, CCSD(T) \\
 &  & [\citenum{Zaremba-Kopczyk2021}] & $^1S_0, 4d^{10}$ & $45.8$ & 2021 & ECP, CCSD(T) \\
 &  & [\citenum{Zhou2021a}] & $^1S_0, 4d^{10}$ & $46 \pm 2$ & 2021 & R, DFCP+RCI \\
 &  & [\citenum{Hohm2022a}] & $^1S_0, 4d^{10}$ & $47.5 \pm 2.0$ & 2022 & exp. \\
49 & In & [\citenum{Guella1984}] & $^2P_{1/2}$ & $68.7 \pm 8.1$ & 1984 & exp. \\
 &  & [\citenum{Cernusak2003}] & $^2P_{1/2}$ & $68.67 \pm 0.69$ & 2003 & R, DK, CCSD(T) \\
 &  & [\citenum{Chu2004}] & $^2P_{1/2}$ & $70.3$ & 2004 & SIC-DFT \\
 &  & [\citenum{Fleig2005}] & $^2P$ & $66.7$ & 2005 & R, SF, MRCI, $M_L$ res. \\
 &  & [\citenum{Fleig2005}] & $^2P_{1/2}/^2P_{3/2}$ & $61.9 \pm 1.2/69.6 \pm 1.4$ & 2005 & R, Dirac, MRCI, $M_J$ res. \\
 &  & [\citenum{Buchachenko2010c}] & $^2P_{1/2}/^2P_{3/2}$ & $66.4 \pm 5.0/74.4 \pm 8.0$ & 2010 & SI-SOCI, $M_J$ res. \\
 &  & [\citenum{Borschevsky2012}] & $^2P_{1/2}/^2P_{3/2}$ & $62.0 \pm 1.9/69.7 \pm 4.0$ & 2012 & \makecell{R, Dirac, FSCC, $M_J$ res. \\($J = 3/2$: $M_J = 3/2$: 55.1, \\$M_J = 1/2$: 84.6)} \\
 &  & [\citenum{Fleig2005, Borschevsky2012}] & $^2P_{1/2}$ & $61.5 \pm 5.6$ & 2012 & CCSD(T) \\
 &  & [\citenum{Safronova2013}] & $^2P_{1/2}$ & $62.4$ & 2013 & R, Dirac + Breit, CI + all-order \\
 &  & [\citenum{Ma2015}] & $^2P_{1/2}$ & $62.1 \pm 6.1$ & 2015 & exp. \\
 &  & [\citenum{Gould2016b}] & $^2P_{1/2}$ & $73.1$ & 2016 & SIC-DFT (RXH) \\
 &  & [\citenum{gobre2016efficient}] & $^2P_{1/2}$ & $70.22$ & 2016 & LR-CCSD \\
 &  & [\citenum{Schwerdtfeger2019}][155] & $^2P_{1/2}$ & $65.2$ & 2019 & R, DFT \\
 &  & [\citenum{A.Manz2019}] & $^2P_{1/2}$ & $67.9$ & 2019 & ECP, CCSD \\
 &  & [\citenum{Schwerdtfeger2019}] & $--$ & $65 \pm 4$ & 2019 & recommended \\
 &  & [\citenum{Neto2021}] & $^2P_{1/2}$ & $70.070$ & 2021 & R, CCSD(T) \\
 &  & [\citenum{Kumar2022}] & $^2P_{1/2}$ & $64.3 \pm 1.3/82.3 \pm 1.7$ & 2022 & R, Breit+QED, CCSD \\
 &  & [\citenum{A.Manz2019}][2023185] & $^2P_{1/2}$ & $64.5$ & 2023 & R, (D)BSR \\
50 & Sn & [\citenum{Lide2004, Doolen1987}] & $^3P$ & $52 \pm 13$ & 2004 & R, Dirac, LDA \\
 &  & [\citenum{Chu2004}] & $^3P$ & $57.5$ & 2004 & SIC-DFT \\
 &  & [\citenum{Thierfelder2008}] & $^3P_0$ & $52.9 \pm 2.1$ & 2008 & R, Dirac + Gaunt, CCSD(T) \\
 &  & [\citenum{Thierfelder2008}] & $^3P_0$ & $42.4 \pm 11$ & 2008 & exp. \\
 &  & [\citenum{Assadollahzadeh2009}] & $^3P_0$ & $54.48$ & 2009 & R, PP, DFT, BP386 \\
 &  & [\citenum{Ma2015}] & $^3P_0$ & $67.5 \pm 8.8$ & 2015 & exp. \\
 &  & [\citenum{Gould2016b}] & $^3P_0$ & $57.9$ & 2016 & SIC-DFT (RXH) \\
 &  & [\citenum{Chu2004, Gould2016a}] & $^3P_0$ & $60.0$ & 2016 & TD-DFT (LEXX) \\
 &  & [\citenum{gobre2016efficient}] & $^3P_0$ & $55.95$ & 2016 & LR-CCSD \\
 &  & [\citenum{A.Manz2019}] & $^3P_0$ & $54.29$ & 2019 & ECP, CCSD \\
 &  & [\citenum{Schwerdtfeger2019}] & $--$ & $53 \pm 6$ & 2019 & recommended \\
 &  & [\citenum{CanalNeto2021}] & $^3P_0$ & $60.115$ & 2021 & R, CCSD(T) \\
 &  & [\citenum{Neto2021}] & $^3P_0$ & $61.063$ & 2021 & R, CCSD(T) \\
51 & Sb & [\citenum{Lide2004, Doolen1987}] & $^4S$ & $45 \pm 11$ & 2004 & R, Dirac, LDA \\
 &  & [\citenum{Roos2004}] & $^4S$ & $42.2 \pm 1.3$ & 2004 & R, DK, CASPT2 \\
 &  & [\citenum{Chu2004}] & $^4S$ & $47.07$ & 2004 & SIC-DFT \\
 &  & [\citenum{Maroulis2007}] & $^4S$ & $42.26$ & 2007 & NR, CCSD(T) \\
 &  & [\citenum{Buchachenko2010}] & $^4S$ & $43.03$ & 2010 & ECP, CCSD(T) \\
 &  & [\citenum{Gould2016b}] & $^4S$ & $45.7$ & 2016 & SIC-DFT (RXH) \\
 &  & [\citenum{Chu2004, Gould2016a}] & $^4S$ & $44.0$ & 2016 & TD-DFT (LEXX) \\
 &  & [\citenum{A.Manz2019}] & $^4S$ & $42.78$ & 2019 & ECP, CCSD \\
 &  & [\citenum{Schwerdtfeger2019}] & $--$ & $43 \pm 2$ & 2019 & recommended \\
 &  & [\citenum{Visentin2020}] & $^4S$ & $42.8$ & 2020 & ECP, CCSD(T) \\
52 & Te & [\citenum{Lide2004, Doolen1987}] & $^3P$ & $37 \pm 4$ & 2004 & R, LDA \\
 &  & [\citenum{Chu2004}] & $^3P$ & $40.06$ & 2004 & SIC-DFT \\
 &  & [\citenum{Maroulis2006, Sadlej1992}] & $^3P$ & $38.1 \pm 3.8$ & 2006 & QR, MVD-HF, GTO basis set \\
 &  & [\citenum{Gould2016b}] & $^3P$ & $36.9$ & 2016 & SIC-DFT (RXH) \\
 &  & [\citenum{gobre2016efficient}] & $^3P$ & $37.65$ & 2016 & LR-CCSD \\
 &  & [\citenum{A.Manz2019}] & $^3P$ & $37.51$ & 2019 & ECP, CCSD \\
 &  & [\citenum{Schwerdtfeger2019}] & $--$ & $38 \pm 4$ & 2019 & recommended \\
 &  & [\citenum{Allehabi2022}] & $^3P$ & $37.3$ & 2022 & R, Dirac, CIPT+SD (RPA) \\
53 & I & [\citenum{Maroulis1997}] & $^2P_{3/2}$ & $32.9 \pm 1.3$ & 1997 & exp. \\
 &  & [\citenum{Maroulis1997}][162] & $^2P_{3/2}$ & $33.4$ & 1997 & exp. \\
 &  & [\citenum{Fleig2002}] & $^2P_{1/2}$ & $35.1$ & 2002 & R, DK, SO-CI \\
 &  & [\citenum{Fleig2002}] & $^2P_{3/2}$ & $34.1$ & 2002 & R, DK, SO-CI, $M_J$ res. \\
 &  & [\citenum{Chu2004}] & $^2P$ & $33.6$ & 2004 & SIC-DFT \\
 &  & [\citenum{Maroulis2006, Fleig2002, Sadlej1992}] & $^2P_{3/2}$ & $33.0 \pm 1.7$ & 2006 & R, DK, SO-CI \\
 &  & [\citenum{Gould2016b}] & $^2P$ & $30.5$ & 2016 & SIC-DFT (RXH) \\
 &  & [\citenum{gobre2016efficient}] & $^2P$ & $35.00$ & 2016 & LR-CCSD \\
 &  & [\citenum{A.Manz2019}] & $^2P$ & $31.57$ & 2019 & ECP, CCSD \\
 &  & [\citenum{Schwerdtfeger2019}] & $--$ & $32.9 \pm 1.3$ & 2019 & recommended \\
 &  & [\citenum{CanalNeto2021}] & $^2P$ & $31.101$ & 2021 & R, CCSD(T) \\
 &  & [\citenum{Neto2021}] & $^2P$ & $31.114$ & 2021 & R, CCSD(T) \\
54 & Xe & [\citenum{Langhoff1969}] & $^1S_0$ & $27.342$ & 1969 & exp. \\
 &  & [\citenum{Huot1991}] & $^1S_0$ & $27.078 \pm 0.050$ & 1991 & exp. \\
 &  & [\citenum{Thakkar1992}] & $^1S_0$ & $27.16$ & 1992 & DOSD (constrained dipole oscillator strength distribution) \\
 &  & [\citenum{Dalgarno1997b}] & $^1S_0$ & $27.292$ & 1997 & exp. \\
 &  & [\citenum{Runeberg1998}] & $^1S_0$ & $27.36$ & 1998 & R, SOPP, CCSD(T) + MP2 basis set correction \\
 &  & [\citenum{Nakajima2001}] & $^1S$ & $27.06 \pm 0.27$ & 2001 & R, DK3, CCSD(T) \\
 &  & [\citenum{Soldán2001a}] & $^1S_0$ & $27.2937 \pm 0.0003$ & 2001 & CCSD(T), ECP \\
 &  & [\citenum{Roos2004}] & $^1S$ & $26.7$ & 2004 & R, DK, CASPT2 \\
 &  & [\citenum{Mani2009}] & $^1S_0$ & $25.297$ & 2009 & R, Dirac, CCSD/T \\
 &  & [\citenum{Bezchastnov2010}] & $^1S_0$ & $27.42$ & 2010 & R, DK3, CCSD(T) \\
 &  & [\citenum{Chattopadhyay2012}] & $^1S_0$ & $26.432$ & 2012 & R, DK3, CCSD \\
 &  & [\citenum{Dzuba2016b}] & $^1S_0$ & $26.7$ & 2016 & R, RPA, PolPot \\
 &  & [\citenum{Gould2016b}] & $^1S_0$ & $25.4$ & 2016 & SIC-DFT (RXH) \\
 &  & [\citenum{gobre2016efficient}] & $^1S_0$ & $27.30$ & 2016 & LR-CCSD \\
 &  & [\citenum{Sahoo2018}] & $^1S_0$ & $28.4 \pm 0.5$ & 2018 & R, CCSD(T) \\
 &  & [\citenum{Sakurai2018}] & $^1S_0$ & $27.508$ & 2018 & R, CCSD(T) \\
 &  & [\citenum{A.Manz2019}] & $^1S_0$ & $26.6$ & 2019 & ECP, CCSD \\
 &  & [\citenum{Schwerdtfeger2019}] & $--$ & $27.32 \pm 0.20$ & 2019 & recommended \\
 &  & [\citenum{Dutta2020}] & $^1S_0$ & $27.099$ & 2020 & R, DHF, MBPT \\
 &  & [\citenum{Sahoo2023}] & $^1S_0$ & $27.55 \pm 0.30$ & 2023 & CCSD + hyperfine-induced contributions \\
55 & Cs & [\citenum{Derevianko1999}] & $^2S_{1/2}$ & $399.9 \pm 1.9$ & 1999 & R, Dirac, SD, all orders + exp. data \\
 &  & [\citenum{Derevianko1999}] & $^2S_{1/2}$ & $401.5$ & 1999 & R, SD all orders + exp. data for electronic transitions \\
 &  & [\citenum{Amini2003}] & $^2S_{1/2}$ & $401.0 \pm 0.6$ & 2003 & exp. \\
 &  & [\citenum{Safronova2004}] & $^2S_{1/2}$ & $398.2 \pm 0.9$ & 2004 & R, Dirac, SDpT \\
 &  & [\citenum{Lim2005}] & $^2S$ & $396.0 \pm 5.9$ & 2005 & R, DK, CCSD(T), AE \\
 &  & [\citenum{Iskrenova-Tchoukova2008a}] & $^2S_{1/2}$ & $398.4 \pm 0.7$ & 2008 & R, DF, RPA, SD-all order \\
 &  & [\citenum{Derevianko2010}] & $^2S_{1/2}$ & $399.8$ & 2010 & Combination of theoretical and experimental data \\
 &  & [\citenum{Borschevsky2013}] & $^2S_{1/2}$ & $399.0$ & 2013 & R, Dirac, CCSD(T) \\
 &  & [\citenum{Jiang2015a}] & $^2S_{1/2}$ & $396.7 \pm 7.9$ & 2015 & Combination of theoretical and experimental data \\
 &  & [\citenum{Singh2016}] & $^2S_{1/2}$ & $399.5 \pm 0.8$ & 2016 & R, Dirac, RCC-SD \\
 &  & [\citenum{Gregoire2015, Gregoire2016}] & $^2S_{1/2}$ & $400.8 \pm 0.4$ & 2016 & exp. \\
 &  & [\citenum{Schwerdtfeger2019}] & $--$ & $400.9 \pm 0.7$ & 2019 & recommended \\
 &  & [\citenum{Smialkowski2021}] & $^2S_{1/2}$ & $391.1$ & 2021 & SR, CCSD(T), ECP \\
 &  & [\citenum{Kaur2022}] & $^2S_{1/2}$ & $399.9 \pm 0.6$ & 2022 & R, DHF, all order \\
 &  & [\citenum{Badhan2022}] & $^2S_{1/2}$ & $399.74 \pm 0.55$ & 2022 & R, Dirac-HF, pertubative singles + doubles method, RPA \\
56 & Ba & [\citenum{Schwartz1974}] & $^1S_0$ & $268 \pm 22$ & 1974 & exp. \\
 &  & [\citenum{Patil2000}] & $^1S_0$ & $261.2$ & 2000 & Model potential \\
 &  & [\citenum{Porsev2006, Porsev2002}] & $^1S$ & $262.2$ & 2006 & R, CI, MBPT \\
 &  & [\citenum{Maroulis2006, Lim2004}] & $^1S_0$ & $273.5 \pm 4.1$ & 2006 & R, DK + SO, CCSD(T) \\
 &  & [\citenum{Porsev2006}] & $^1S_0$ & $272.1$ & 2006 & Hybrid-RCI + MBPT sum rule \\
 &  & [\citenum{Schäfer2007}] & $^1S_0$ & $275.5 \pm 5.5$ & 2007 & R, DK, CCSD(T) \\
 &  & [\citenum{Sahoo2008}] & $^1S_0$ & $268.19$ & 2008 & R, Dirac, coupled cluster \\
 &  & [\citenum{Porsev2006, Derevianko2010}] & $^1S_0$ & $273.5 \pm 2.0$ & 2010 & Hybrid-RCI + MBPT sum rule, recommended \\
 &  & [\citenum{Borschevsky2013a}] & $^1S_0$ & $272.7$ & 2013 & R, Dirac + Gaunt, CCSD(T) \\
 &  & [\citenum{Chattopadhyay2014}] & $^1S_0$ & $274.68$ & 2014 & \makecell{R, Dirac + Breit, perturbed relativistic \\coupled-cluster theory (PRCC)} \\
 &  & [\citenum{Jiang2015a}] & $^1S_0$ & $278.1 \pm 5.6$ & 2015 & Combination of theoretical and experimental data \\
 &  & [\citenum{Dzuba2016b}] & $^1S_0$ & $251$ & 2016 & R, RPA, PolPot \\
 &  & [\citenum{gobre2016efficient}] & $^1S_0$ & $275.0$ & 2016 & LR-CCSD \\
 &  & [\citenum{Buchachenko2018}] & $^1S_0$ & $274.92 \pm 0.01$ & 2018 & CCSD(T), R DKH ECP/CBS \\
 &  & [\citenum{Visentin2019}] & $^1S_0$ & $273.90/276.98$ & 2019 & CCSD(T), R, ECP-46/X2C-46 \\
 &  & [\citenum{Bala2019}] & $^1S_0$ & $269.0$ & 2019 & R, KRCISD/aug-QZ \\
 &  & [\citenum{Schwerdtfeger2019}] & $--$ & $272 \pm 10$ & 2019 & recommended \\
 &  & [\citenum{Shukla2020}] & $^1S_0$ & $276.2$ & 2020 & R, MCDF \\
57 & La & [\citenum{Lide2004, Doolen1987}] & $^2D_{3/2}, 5d^1$ & $210 \pm 52$ & 2004 & R, Dirac, LDA \\
 &  & [\citenum{Chu2007}] & $^2D_{3/2}, 5d^1$ & $201 \pm 40$ & 2007 & TD-DFT \\
 &  & [\citenum{Buchachenko2007}] & $^2D_{3/2}, 5d^1$ & $219.8$ & 2007 & R, CASSCF, ECP \\
 &  & [\citenum{Hohm2012, Buchachenko2007}] & $^2D_{3/2}, 5d^1$ & $220 \pm 22$ & 2012 & R, CASSCF, ECP \\
 &  & [\citenum{Dzuba2014}] & $^2D_{3/2}, 5d^1$ & $213.7$ & 2014 & \makecell{R, Dirac, CI + MBPT + CP(RPA); \\($\alpha_D$ = 218.7 for the $5d^2 6s^1$ configuration)} \\
 &  & [\citenum{Ma2015}] & $^2D_{3/2}, 5d^1$ & $170.7 \pm 8.1$ & 2015 & exp. \\
 &  & [\citenum{A.Manz2019}] & $^2D_{3/2}, 5d^1$ & $214.72$ & 2019 & ECP, CCSD \\
 &  & [\citenum{Schwerdtfeger2019}] & $--$ & $215 \pm 20$ & 2019 & recommended \\
 &  & [\citenum{Centoducatte2022}] & $^2D_{3/2}, 5d^1$ & $190.9$ & 2022 & R (ZORA), DFT (B3LYP) \\
58 & Ce & [\citenum{Lide2004, Doolen1987}] & $4f^1 5d^1$ & $200 \pm 50$ & 2004 & R, Dirac, LDA \\
 &  & [\citenum{Chu2007}] & $4f^1 5d^1$ & $194 \pm 39$ & 2007 & TD-DFT \\
 &  & [\citenum{Dzuba2014}] & $4f^1 5d^1$ & $204.7$ & 2014 & \makecell{R, Dirac, CI + MBPT + CP(RPA); \\($\alpha_D$ = 223.4 for the $4f^2$ configuration)} \\
 &  & [\citenum{Ma2015}] & $^1G_4, 4f^1 5d^1$ & $192 \pm 20$ & 2015 & exp. \\
 &  & [\citenum{Schwerdtfeger2019}] & $--$ & $205 \pm 20$ & 2019 & recommended \\
 &  & [\citenum{Ferreira2020}] & $4f^1 5d^1$ & $206.51$ & 2020 & R, DKH2-B3LYP/ADZP \\
 &  & [\citenum{Ferreira2020}] & $4f^1 5d^1$ & $219.66$ & 2020 & R, DKH2-B3LYP/ADZP-DKH \\
59 & Pr & [\citenum{Lide2004, Doolen1987}] & $4f^3$ & $190 \pm 48$ & 2004 & R, Dirac, LDA \\
 &  & [\citenum{Chu2007}] & $4f^3$ & $220 \pm 44$ & 2007 & TD-DFT \\
 &  & [\citenum{Dzuba2014}] & $4f^3$ & $215.8$ & 2014 & \makecell{R, Dirac, CI + MBPT + CP(RPA); \\($\alpha_D$ = 195.7 for the $4f^2 5d^1$ configuration)} \\
 &  & [\citenum{Ma2015}] & $^4I_{9/2}, 4f^3$ & $239 \pm 28$ & 2015 & exp. \\
 &  & [\citenum{Schwerdtfeger2019}] & $--$ & $216 \pm 20$ & 2019 & recommended \\
60 & Nd & [\citenum{Lide2004, Doolen1987}] & $4f^4$ & $212 \pm 53$ & 2004 & R, Dirac, LDA \\
 &  & [\citenum{Chu2007}] & $4f^4$ & $213 \pm 43$ & 2007 & TD-DFT \\
 &  & [\citenum{Dzuba2014}] & $4f^4$ & $208.4$ & 2014 & \makecell{R, Dirac, CI + MBPT + CP(RPA); \\($\alpha_D$ = 187.5 for the $4f^3 5d^1$ configuration)} \\
 &  & [\citenum{Ma2015}] & $^5I_4, 4f^4$ & $184 \pm 20$ & 2015 & exp. \\
 &  & [\citenum{Schwerdtfeger2019}] & $--$ & $208 \pm 20$ & 2019 & recommended \\
 &  & [\citenum{Ferreira2020}] & $4f^4$ & $194.56$ & 2020 & R, DKH2-B3LYP/ADZP \\
 &  & [\citenum{Ferreira2020}] & $4f^4$ & $203.07$ & 2020 & R, DKH2-B3LYP/ADZP-DKH \\
61 & Pm & [\citenum{Lide2004, Doolen1987}] & $4f^5$ & $203 \pm 51$ & 2004 & R, Dirac, LDA \\
 &  & [\citenum{Chu2007}] & $4f^5$ & $206 \pm 41$ & 2007 & TD-DFT \\
 &  & [\citenum{Dzuba2014}] & $4f^5$ & $200.2$ & 2014 & \makecell{R, Dirac, CI + MBPT + CP(RPA); \\($\alpha_D$ = 179.3 for the $4f^4 5d^1$ configuration)} \\
 &  & [\citenum{Schwerdtfeger2019}] & $--$ & $200 \pm 20$ & 2019 & recommended \\
62 & Sm & [\citenum{Lide2004, Doolen1987}] & $4f^6$ & $194 \pm 48$ & 2004 & R, Dirac, LDA \\
 &  & [\citenum{Chu2007}] & $4f^6$ & $200 \pm 40$ & 2007 & TD-DFT \\
 &  & [\citenum{Buchachenko2007}] & $4f^6$ & $196.8$ & 2007 & R, CASSCF, ECP \\
 &  & [\citenum{Hohm2012, Buchachenko2007}] & $4f^6$ & $197 \pm 20$ & 2012 & R, CASSCF, ECP \\
 &  & [\citenum{Dzuba2014}] & $4f^6$ & $192.1$ & 2014 & \makecell{R, Dirac, CI + MBPT + CP(RPA); \\($\alpha_D$ = 171.7 for the $4f^5 5d^1$ configuration)} \\
 &  & [\citenum{Ma2015}] & $^7F_0, 4f^6$ & $157 \pm 16$ & 2015 & exp. \\
 &  & [\citenum{Schwerdtfeger2019}] & $--$ & $192 \pm 20$ & 2019 & recommended \\
63 & Eu & [\citenum{Lide2004, Doolen1987}] & $4f^7$ & $187 \pm 47$ & 2004 & R, Dirac, LDA \\
 &  & [\citenum{Chu2007}] & $4f^7$ & $194 \pm 39$ & 2007 & TD-DFT \\
 &  & [\citenum{Buchachenko2007}] & $4f^7$ & $189.4$ & 2007 & R, CASSCF, ECP \\
 &  & [\citenum{Hohm2012, Buchachenko2007}] & $4f^7$ & $189 \pm 19$ & 2012 & R, CASSCF, ECP \\
 &  & [\citenum{Dzuba2014}] & $4f^7$ & $184.2$ & 2014 & \makecell{R, Dirac, CI + MBPT + CP(RPA); \\($\alpha_D$ = 164.7 for the $4f^6 5d^1$ configuration)} \\
 &  & [\citenum{Ma2015}] & $^8S_{7/2}, 4f^7$ & $155 \pm 25$ & 2015 & exp. \\
 &  & [\citenum{Schwerdtfeger2019}] & $--$ & $184 \pm 20$ & 2019 & recommended \\
 &  & [\citenum{Flambaum2020}] & $4f^7$ & $188$ & 2020 & r, CI+MBPT \\
64 & Gd & [\citenum{Lide2004, Doolen1987}] & $4f^7 5d^1$ & $159 \pm 40$ & 2004 & R, Dirac, LDA \\
 &  & [\citenum{Chu2007}] & $4f^7 5d^1$ & $161 \pm 32$ & 2007 & TD-DFT \\
 &  & [\citenum{Dzuba2014}] & $4f^7 5d^1$ & $158.3$ & 2014 & \makecell{R, Dirac, CI + MBPT + CP(RPA); \\($\alpha_D$ = 194.5 for the $4f^7 5d^2 6s^1$ configuration)} \\
 &  & [\citenum{Ma2015}] & $^9D_2, 4f^7 5d^1$ & $176 \pm 26$ & 2015 & exp. \\
 &  & [\citenum{Schwerdtfeger2019}] & $--$ & $158 \pm 20$ & 2019 & recommended \\
 &  & [\citenum{Flambaum2020}] & $4f^7 5d^1$ & $159$ & 2020 & R, CI+MBPT \\
 &  & [\citenum{Ferreira2020}] & $4f^7 5d^1$ & $171.40$ & 2020 & R, DKH2-B3LYP/ADZP \\
 &  & [\citenum{Ferreira2020}] & $4f^7 5d^1$ & $145.74$ & 2020 & R, DKH2-B3LYP/ADZP-DKH \\
65 & Tb & [\citenum{Lide2004, Doolen1987}] & $4f^9$ & $172 \pm 43$ & 2004 & R, Dirac, LDA \\
 &  & [\citenum{Chu2007}] & $4f^9$ & $181 \pm 36$ & 2007 & TD-DFT \\
 &  & [\citenum{Dzuba2014}] & $4f^9$ & $169.5$ & 2014 & \makecell{R, Dirac, CI + MBPT + CP(RPA); \\($\alpha_D$ = 152.4 for the $4f^8 5d^1$ configuration)} \\
 &  & [\citenum{Ma2015}] & $^6H_{15/2}, 4f^9$ & $159 \pm 11$ & 2015 & exp. \\
 &  & [\citenum{Schwerdtfeger2019}] & $--$ & $170 \pm 20$ & 2019 & recommended \\
66 & Dy & [\citenum{Lide2004, Doolen1987}] & $4f^{10}$ & $165 \pm 41$ & 2004 & R, Dirac, LDA \\
 &  & [\citenum{Chu2007}] & $4f^{10}$ & $175 \pm 35$ & 2007 & TD-DFT \\
 &  & [\citenum{Dzuba2014}] & $4f^{10}$ & $162.7$ & 2014 & \makecell{R, Dirac, CI + MBPT + CP(RPA); \\($\alpha_D$ = 148.3 for the $4f^9 5d^1$ configuration)} \\
 &  & [\citenum{Dzuba2014}] & $4f^{10}$ & $165$ & 2014 & R, RPA, PolPot \\
 &  & [\citenum{Ma2015}] & $^5I_8, 4f^{10}$ & $157 \pm 11$ & 2015 & exp. \\
 &  & [\citenum{Dzuba2016b}] & $4f^{10}$ & $168$ & 2016 & R, RPA, PolPot \\
 &  & [\citenum{Li2016}] & $^5I_8, 4f^{10}$ & $164$ & 2016 & exp. \\
 &  & [\citenum{Schwerdtfeger2019}] & $--$ & $163 \pm 15$ & 2019 & recommended \\
 &  & [\citenum{Flambaum2020}] & $4f^{10}$ & $164$ & 2020 & R, CI+MBPT \\
 &  & [\citenum{Ferreira2020}] & $4f^{10}$ & $169.69$ & 2020 & R, DKH2-B3LYP/ADZP \\
 &  & [\citenum{Ferreira2020}] & $4f^{10}$ & $157.20$ & 2020 & R, DKH2-B3LYP/ADZP-DKH \\
67 & Ho & [\citenum{Lide2004, Doolen1987}] & $4f^{11}$ & $159 \pm 40$ & 2004 & R, Dirac, LDA \\
 &  & [\citenum{Chu2007}] & $4f^{11}$ & $170 \pm 34$ & 2007 & TD-DFT \\
 &  & [\citenum{Dzuba2014}] & $4f^{11}$ & $156.3$ & 2014 & \makecell{R, Dirac, CI + MBPT + CP(RPA); \\($\alpha_D$ = 142.9 for the $4f^{{10}} 5d^1$ configuration)} \\
 &  & [\citenum{Ma2015}] & $^4I_{15/2}, 4f^{11}$ & $145 \pm 12$ & 2015 & exp. \\
 &  & [\citenum{Dzuba2016b}] & $4f^{11}$ & $161$ & 2016 & R, RPA, PolPot \\
 &  & [\citenum{Li2016}] & $^4I_{15/2}, 4f^{11}$ & $160$ & 2016 & exp. \\
 &  & [\citenum{Schwerdtfeger2019}] & $--$ & $156 \pm 10$ & 2019 & recommended \\
68 & Er & [\citenum{Lide2004, Doolen1987}] & $4f^{12}$ & $153 \pm 38$ & 2004 & R, Dirac, LDA \\
 &  & [\citenum{Chu2007}] & $4f^{12}$ & $166 \pm 33$ & 2007 & TD-DFT \\
 &  & [\citenum{Dzuba2014}] & $4f^{12}$ & $150.2$ & 2014 & \makecell{R, Dirac, CI + MBPT + CP(RPA); \\($\alpha_D$ = 139.4 for the $4f^{11} 5d^1$ configuration)} \\
 &  & [\citenum{Dzuba2014}] & $4f^{12}$ & $169$ & 2014 & R, RPA, PolPot \\
 &  & [\citenum{Lepers2014}] & $4f^{12}$ & $141 \pm 7$ & 2014 & R, HF, Darwin, SO \\
 &  & [\citenum{Ma2015}] & $^3H_6, 4f^{12}$ & $217 \pm 39$ & 2015 & exp. \\
 &  & [\citenum{Dzuba2016b}] & $4f^{12}$ & $154$ & 2016 & R, RPA, PolPot \\
 &  & [\citenum{Becher2018}] & $4f^{12}$ & $149$ & 2018 & R, HF, Darwin, SO \\
 &  & [\citenum{Becher2018}] & $^3H_6, 4f^{12}$ & $155$ & 2018 & exp. \\
 &  & [\citenum{Schwerdtfeger2019}] & $--$ & $150 \pm 10$ & 2019 & recommended \\
 &  & [\citenum{Ferreira2020}] & $4f^{12}$ & $143.98$ & 2020 & R, DKH2-B3LYP/ADZP \\
 &  & [\citenum{Ferreira2020}] & $4f^{12}$ & $145.01$ & 2020 & R, DKH2-B3LYP/ADZP-DKH \\
 &  & [\citenum{Dzuba2020}] & $^3H_6, 4f^{12}$ & $166.67$ & 2020 & R, Dirac, CIPT+HF+RPA \\
69 & Tm & [\citenum{Lide2004, Doolen1987}] & $4f^{13}$ & $147 \pm 37$ & 2004 & R, Dirac, LDA \\
 &  & [\citenum{Chu2007}] & $4f^{13}$ & $161 \pm 32$ & 2007 & TD-DFT \\
 &  & [\citenum{Buchachenko2007}] & $4f^{13}$ & $152.2$ & 2007 & R, CASSCF, ECP \\
 &  & [\citenum{Hohm2012, Buchachenko2006}] & $4f^{13}$ & $152 \pm 15$ & 2012 & R, MR-ACQQ, ECP \\
 &  & [\citenum{Dzuba2014}] & $4f^{13}$ & $144.3$ & 2014 & \makecell{R, Dirac, CI + MBPT + CP(RPA); \\($\alpha_D$ = 137.8 for the $4f^{{12}} 5d^1$ configuration)} \\
 &  & [\citenum{Ma2015}] & $^2F_{7/2}, 4f^{13}$ & $130 \pm 16$ & 2015 & exp. \\
 &  & [\citenum{Dzuba2016b}] & $4f^{13}$ & $147$ & 2016 & R, RPA, PolPot \\
 &  & [\citenum{Schwerdtfeger2019}] & $--$ & $144 \pm 15$ & 2019 & recommended \\
 &  & [\citenum{Dzuba2020}] & $^2F^0_{7/2}, 4f^{13}$ & $153.02$ & 2020 & R, Dirac, CIPT+HF+RPA \\
70 & Yb & [\citenum{Wang1998}] & $^1S_0, 4f^{14}$ & $141 \pm 4$ & 1998 & R, DHF + Breit + QED, PP \\
 &  & [\citenum{Lide2004, Doolen1987}] & $^1S_0, 4f^{14}$ & $142 \pm 36$ & 2004 & R, Dirac, LDA \\
 &  & [\citenum{Buchachenko2006}] & $^1S_0, 4f^{14}$ & $152.9$ & 2006 & R, Dirac, CCSD(T) \\
 &  & [\citenum{Zhang2007b}] & $^1S_0, 4f^{14}$ & $143$ & 2007 & R, DCHF, CCSD(T), ECP \\
 &  & [\citenum{Chu2007}] & $^1S_0, 4f^{14}$ & $157.3$ & 2007 & TD-DFT \\
 &  & [\citenum{Buchachenko2007}] & $^1S_0, 4f^{14}$ & $151.0$ & 2007 & R, CASSCF, ECP \\
 &  & [\citenum{Sahoo2008}] & $^1S_0, 4f^{14}$ & $144.6 \pm 5.6$ & 2008 & R, Dirac, coupled cluster \\
 &  & [\citenum{Thierfelder2009}] & $^1S_0, 4f^{14}$ & $140.7 \pm 7.0$ & 2009 & R, Dirac + Gaunt, CCSD(T) \\
 &  & [\citenum{Zhang2009}] & $^1S_0, 4f^{14}$ & $144$ & 2009 & R, CCSD, PolPot \\
 &  & [\citenum{Thierfelder2009}] & $^1S_0, 4f^{14}$ & $140.44$ & 2009 & R, Dirac, CCSD(T) \\
 &  & [\citenum{Buchachenko2010d}] & $^1S_0, 4f^{14}$ & $142.6$ & 2010 & ECP, CCSD(T) \\
 &  & [\citenum{Dammalapati2011}] & $^1S_0, 4f^{14}$ & $141 \pm 6$ & 2011 & \makecell{R, Dirac, CI + MBPT + experimental data, \\see also ref [\citenum{{Beloy2012}}] for error estimates} \\
 &  & [\citenum{Safronova2012}] & $^1S_0, 4f^{14}$ & $141 \pm 2$ & 2012 & R, Dirac, CI + MBPT + RPA \\
 &  & [\citenum{Hohm2012, Buchachenko2006}] & $^1S_0, 4f^{14}$ & $145.3 \pm 4.4$ & 2012 & R, Dirac, CCSD(T) \\
 &  & [\citenum{Beloy2012}] & $^1S_0, 4f^{14}$ & $139.3 \pm 5.9$ & 2012 & exp. \\
 &  & [\citenum{Dzuba2014}] & $^1S_0, 4f^{14}$ & $138.9$ & 2014 & \makecell{R, Dirac, CI + MBPT + CP(RPA); \\($\alpha_D$ = 312.2 for \\the $4f^14 6s^1 6p^1$ configuration)} \\
 &  & [\citenum{Ma2015}] & $^1S_0, 4f^{14}$ & $147 \pm 20$ & 2015 & exp. \\
 &  & [\citenum{Dzuba2016b}] & $^1S_0, 4f^{14}$ & $142$ & 2016 & R, RPA, PolPot \\
 &  & [\citenum{Yoshizawa2016}] & $^1S_0, 4f^{14}$ & $135.73$ & 2016 & R, DFT, CAM-B3LYP, 2c-NESC \\
 &  & [\citenum{Yoshizawa2016}] & $^1S_0, 4f^{14}$ & $147.26$ & 2016 & R, DFT, PBE0, 2c-NESC \\
 &  & [\citenum{Sahoo2017}] & $^1S_0, 4f^{14}$ & $135.50$ & 2017 & R, CCSD \\
 &  & [\citenum{Sahoo2018}] & $^1S_0, 4f^{14}$ & $136 \pm 5$ & 2018 & R, CCSD(T) \\
 &  & [\citenum{Tang2018}] & $^1S_0, 4f^{14}$ & $135 \pm 3$ & 2018 & R, CI+MBPT+FC \\
 &  & [\citenum{Dzuba2018}] & $^1S_0, 4f^{14}$ & $150 \pm 9$ & 2018 & R, CIPT \\
 &  & [\citenum{Visentin2019}] & $^1S_0, 4f^{14}$ & $140.54$ & 2019 & R, CCSD(T) \\
 &  & [\citenum{Schwerdtfeger2019}] & $--$ & $139 \pm 6$ & 2019 & recommended \\
 &  & [\citenum{Flambaum2020}] & $^1S_0, 4f^{14}$ & $147$ & 2020 & R, CI+MBPT \\
 &  & [\citenum{Ferreira2020}] & $^1S_0, 4f^{14}$ & $133.65$ & 2020 & R, DKH2-B3LYP/ADZP \\
 &  & [\citenum{Ferreira2020}] & $^1S_0, 4f^{14}$ & $134.44$ & 2020 & R, DKH2-B3LYP/ADZP \\
 &  & [\citenum{Dzuba2020}] & $^1S_0, 4f^{14}$ & $143$ & 2020 & R, Dirac, CIPT+HF+RPA \\
 &  & [\citenum{Tomza2021}] & $^1S_0, 4f^{14}$ & $136.0$ & 2021 & SR, ECP, CCSD(T) \\
 &  & [\citenum{Tang2023}] & $^1S_0, 4f^{14}$ & $139 \pm 3$ & 2023 & MCDHF+Breit+QED \\
71 & Lu & [\citenum{Lide2004, Doolen1987}] & $^2D_{3/2}, 5d^1$ & $148 \pm 17$ & 2004 & R, Dirac, LDA \\
 &  & [\citenum{Chu2007}] & $^2D_{3/2}, 5d^1$ & $131 \pm 26$ & 2007 & TD-DFT \\
 &  & [\citenum{Dzuba2014}] & $^2D_{3/2}, 5d^1$ & $137 \pm 7$ & 2014 & \makecell{R, Dirac, CI + MBPT + CP(RPA); \\($\alpha_D$ = 61.3 for the \\$4f^{{14}} 6s^2 6p^1$ configuration)} \\
 &  & [\citenum{Dzuba2014a}] & $^2D_{3/2}, 5d^1$ & $145$ & 2014 & R, DF, CI + all-order + Breit + QED \\
 &  & [\citenum{Ma2015}] & $^2D_{3/2}, 5d^1$ & $124 \pm 18$ & 2015 & exp. \\
 &  & [\citenum{Schwerdtfeger2019}] & $--$ & $137 \pm 7$ & 2019 & recommended \\
72 & Hf & [\citenum{Lide2004, Doolen1987}] & $^3F_2, 5d^2$ & $109 \pm 27$ & 2004 & R, Dirac, LDA \\
 &  & [\citenum{Dzuba2014a}] & $^3F_2, 5d^2$ & $97$ & 2014 & R, DF, CI + all-order + Breit + QED \\
 &  & [\citenum{Dzuba2014, Dzuba2014a}] & $^3F_2, 5d^2$ & $103 \pm 5$ & 2014 & R, DF, CI + MBPT + Breit + QED \\
 &  & [\citenum{Ma2015}] & $^3F_2, 5d^2$ & $84 \pm 19$ & 2015 & exp. \\
 &  & [\citenum{Sadlej1991a, Gould2016a}] & $^3F_2, 5d^2$ & $83.7$ & 2016 & NR, MBPT4 \\
 &  & [\citenum{gobre2016efficient}] & $^3F_2, 5d^2$ & $99.52$ & 2016 & LR-CCSD \\
 &  & [\citenum{A.Manz2019}] & $^3F_2, 5d^2$ & $102.55$ & 2019 & ECP, CCSD \\
 &  & [\citenum{Schwerdtfeger2019}] & $--$ & $103 \pm 6$ & 2019 & recommended \\
 &  & [\citenum{Centoducatte2022}] & $^3F_2, 5d^2$ & $95.6$ & 2022 & R (ZORA), DFT (B3LYP) \\
 &  & [\citenum{Neto2023}] & $^3F_2, 5d^2$ & $94.2$ & 2023 & R (ATZP-ZORA), DFT (B3LYP) \\
73 & Ta & [\citenum{Liepack1956}] & $^4F_{3/2}, 5d^3$ & $115 \pm 20$ & 1956 & exp. \\
 &  & [\citenum{Cole1986}] & $^4F_{3/2}, 5d^3$ & $128 \pm 20$ & 1986 & exp. \\
 &  & [\citenum{Cole1986}] & $^4F_{3/2}, 5d^3$ & $108 \pm 20$ & 1986 & exp. \\
 &  & [\citenum{Lide2004, Doolen1987}] & $^4F_{3/2}, 5d^3$ & $88 \pm 22$ & 2004 & R, Dirac, LDA \\
 &  & [\citenum{Ma2015}] & $^4F_{3/2}, 5d^3$ & $58 \pm 12$ & 2015 & exp. \\
 &  & [\citenum{Dzuba2016b}] & $^4F_{3/2}, 5d^3$ & $73.7$ & 2016 & R, RPA, PolPot \\
 &  & [\citenum{Gould2016a}] & $^4F_{3/2}, 5d^3$ & $73.9$ & 2016 & TD-DFT (LEXX) \\
 &  & [\citenum{gobre2016efficient}] & $^4F_{3/2}, 5d^3$ & $82.53$ & 2016 & LR-CCSD \\
 &  & [\citenum{A.Manz2019}] & $^4F_{3/2}, 5d^3$ & $84.22$ & 2019 & ECP, CCSD \\
 &  & [\citenum{Schwerdtfeger2019}] & $--$ & $74 \pm 20$ & 2019 & recommended \\
 &  & [\citenum{Centoducatte2022}] & $^4F_{3/2}, 5d^3$ & $79.6$ & 2022 & R (ZORA), DFT (B3LYP) \\
74 & W & [\citenum{Liepack1956}] & $^5D_0, 5d^4$ & $47 \pm 7$ & 1956 & exp. \\
 &  & [\citenum{Lide2004, Doolen1987}] & $^5D_0, 5d^4$ & $75 \pm 19$ & 2004 & R, Dirac, LDA \\
 &  & [\citenum{Dzuba2016b}] & $^5d^4$ & $68.1$ & 2016 & R, RPA, PolPot \\
 &  & [\citenum{Gould2016a}] & $^5D_0, 5d^4$ & $65.8$ & 2016 & TD-DFT (LEXX) \\
 &  & [\citenum{gobre2016efficient}] & $^5D_0, 5d^4$ & $68.5$ & 2016 & LR-CCSD \\
 &  & [\citenum{A.Manz2019}] & $^5D_0, 5d^4$ & $71.04$ & 2019 & ECP, CCSD \\
 &  & [\citenum{Schwerdtfeger2019}] & $--$ & $68 \pm 15$ & 2019 & recommended \\
 &  & [\citenum{Centoducatte2022}] & $^5D_0, 5d^4$ & $65 \pm 13$ & 2022 & R (ZORA), DFT (B3LYP) \\
 &  & [\citenum{Centoducatte2022}] & $^5D_0, 5d^4$ & $73.2$ & 2022 & R (ZORA), CCSD(T) \\
 &  & [\citenum{Sarkisov2022}] & $^5D_0, 5d^4$ & $68.98$ & 2022 & exp. \\
75 & Re & [\citenum{Lide2004, Doolen1987}] & $^6S_{5/2}, 5d^5$ & $65 \pm 16$ & 2004 & R, Dirac, LDA \\
 &  & [\citenum{Roos2005}] & $^6S_{5/2}, 5d^5$ & $61.1$ & 2005 & DK, CASPT2 \\
 &  & [\citenum{Buchachenko2010}] & $^6S_{5/2}, 5d^5$ & $61.9$ & 2010 & R, CCSD(T) \\
 &  & [\citenum{Dzuba2016b}] & $^5d^5$ & $65.6$ & 2016 & R, RPA, PolPot \\
 &  & [\citenum{Gould2016a}] & $^6S_{5/2}, 5d^5$ & $60.2$ & 2016 & TD-DFT (LEXX) \\
 &  & [\citenum{gobre2016efficient}] & $^6S_{5/2}, 5d^5$ & $63.04$ & 2016 & LR-CCSD \\
 &  & [\citenum{A.Manz2019}] & $^6S_{5/2}, 5d^5$ & $65.55$ & 2019 & ECP, CCSD \\
 &  & [\citenum{Schwerdtfeger2019}] & $--$ & $62 \pm 3$ & 2019 & recommended \\
76 & Os & [\citenum{Lide2004, Doolen1987}] & $^5D_4, 5d^6$ & $57$ & 2004 & R, Dirac, LDA \\
 &  & [\citenum{Dzuba2016b}] & $^5d^6$ & $57.8$ & 2016 & R, RPA, PolPot \\
 &  & [\citenum{Gould2016a}] & $^5D_4, 5d^6$ & $55.3$ & 2016 & TD-DFT (LEXX) \\
 &  & [\citenum{gobre2016efficient}] & $^5D_4, 5d^6$ & $55.06$ & 2016 & LR-CCSD \\
 &  & [\citenum{A.Manz2019}] & $^5D_4, 5d^6$ & $56.56$ & 2019 & ECP, CCSD \\
 &  & [\citenum{Schwerdtfeger2019}] & $--$ & $57 \pm 3$ & 2019 & recommended \\
 &  & [\citenum{Centoducatte2022}] & $^5D_4, 5d^6$ & $53.1$ & 2022 & R (ZORA), DFT (B3LYP) \\
 &  & [\citenum{Neto2023}] & $^5D_4, 5d^6$ & $54.1$ & 2023 & R (ATZP-ZORA), DFT (B3LYP) \\
77 & Ir & [\citenum{Cole1986, Bardon1984}] & $^4F_{9/2}, 5d^7$ & $54.0 \pm 6.7$ & 1986 & exp. \\
 &  & [\citenum{Lide2004, Doolen1987}] & $^4F_{9/2}, 5d^7$ & $51 \pm 13$ & 2004 & R, Dirac, LDA \\
 &  & [\citenum{Dzuba2016b}] & $^5d^7$ & $51.7$ & 2016 & R, RPA, PolPot \\
 &  & [\citenum{Gould2016a}] & $^4F_{9/2}, 5d^7$ & $51.3$ & 2016 & TD-DFT (LEXX) \\
 &  & [\citenum{Gould2016a}] & $^4F_{9/2}, 5d^7$ & $51.3$ & 2016 & TD-DFT (LEXX) \\
 &  & [\citenum{gobre2016efficient}] & $^4F_{9/2}, 5d^7$ & $42.51$ & 2016 & LR-CCSD \\
 &  & [\citenum{A.Manz2019}] & $^4F_{9/2}, 5d^7$ & $49.48$ & 2019 & ECP, CCSD \\
 &  & [\citenum{Schwerdtfeger2019}] & $--$ & $54 \pm 7$ & 2019 & recommended \\
 &  & [\citenum{Centoducatte2022}] & $^4F_{9/2}, 5d^7$ & $40.0$ & 2022 & R (ZORA), DFT (B3LYP) \\
 &  & [\citenum{Neto2023}] & $^4F_{9/2}, 5d^7$ & $39.8$ & 2023 & R (ATZP-ZORA), DFT (B3LYP) \\
78 & Pt & [\citenum{Lide2004, Doolen1987}] & $^3D_3, 5d^9$ & $44 \pm 11$ & 2004 & R, Dirac, LDA \\
 &  & [\citenum{Gould2016a}] & $^3D_3, 5d^9$ & $48.0$ & 2016 & TD-DFT (LEXX) \\
 &  & [\citenum{gobre2016efficient}] & $^3D_3, 5d^9$ & $39.68$ & 2016 & LR-CCSD \\
 &  & [\citenum{A.Manz2019}] & $^3D_3, 5d^9$ & $43.83$ & 2019 & ECP, CCSD \\
 &  & [\citenum{Schwerdtfeger2019}] & $--$ & $48 \pm 4$ & 2019 & recommended \\
 &  & [\citenum{Irikura2021}] & $^3D_3, 5d^9$ & $41.2 \pm 1.1$ & 2021 & ECP, CCSD(T) \\
 &  & [\citenum{Centoducatte2022}] & $^3D_3, 5d^9$ & $37.5$ & 2022 & R (ZORA), DFT (B3LYP) \\
 &  & [\citenum{Sarkisov2022}] & $^3D_3, 5d^9$ & $38 \pm 8$ & 2022 & exp. \\
 &  & [\citenum{Neto2023}] & $^3D_3, 5d^9$ & $40.9$ & 2023 & R (ATZP-ZORA), DFT (B3LYP) \\
79 & Au & [\citenum{Henderson1997}] & $^2S_{1/2}, 5d^{10}$ & $30 \pm 4$ & 1997 & R, HFR, HS, CI, CACP \\
 &  & [\citenum{Wesendrup2000}] & $^2S, 5d^{10}$ & $34.9$ & 2000 & R, DK, CCSD(T) \\
 &  & [\citenum{Roos2005}] & $^2S_{1/2}, 5d^{10}$ & $39.1 \pm 9.8$ & 2005 & exp. \\
 &  & [\citenum{Maroulis2006, Neogrády1997}] & $^2S, 5d^{10}$ & $36.06 \pm 0.54$ & 2006 & R, DK, CCSD(T) \\
 &  & [\citenum{Schwerdtfeger1994, Mohr2009, Schwerdtfeger2000}] & $^2S, 5d^{10}$ & $35.1$ & 2009 & R, PP, QCISD(T) \\
 &  & [\citenum{Hohm2012, Roos2005}] & $^2S, 5d^{10}$ & $27.9 \pm 4.2$ & 2012 & R, DK, CASPT2 \\
 &  & [\citenum{Hohm2012, Sarkisov2006}] & $^2S_{1/2}, 5d^{10}$ & $49.1 \pm 4.9$ & 2012 & exp. \\
 &  & [\citenum{Gould2016a}] & $^2S_{1/2}, 5d^{10}$ & $45.4$ & 2016 & TD-DFT (LEXX) \\
 &  & [\citenum{gobre2016efficient}] & $^2S_{1/2}, 5d^{10}$ & $36.50$ & 2016 & LR-CCSD \\
 &  & [\citenum{A.Manz2019}] & $^2S_{1/2}, 5d^{10}$ & $39.56$ & 2019 & ECP, CCSD \\
 &  & [\citenum{Schwerdtfeger2019}] & $--$ & $36 \pm 3$ & 2019 & recommended \\
 &  & [\citenum{Dzuba2021}] & $^2S_{1/2}, 5d^{10}$ & $34.0$ & 2021 & R, CI+MBPT \\
 &  & [\citenum{Tomza2021}] & $^2S_{1/2}, 5d^{10}$ & $36.3$ & 2021 & SR, ECP, CCSD(T) \\
 &  & [\citenum{Centoducatte2022}] & $^2S_{1/2}, 5d^{10}$ & $34.2$ & 2022 & R (ZORA), DFT (B3LYP) \\
 &  & [\citenum{Sarkisov2022}] & $^2S_{1/2}, 5d^{10}$ & $40 \pm 8$ & 2022 & exp. \\
 &  & [\citenum{Neto2023}] & $^2S_{1/2}, 5d^{10}$ & $34.1$ & 2023 & R (ATZP-ZORA), DFT (B3LYP) \\
80 & Hg & [\citenum{Kellö1995}] & $^1S, 5d^{10}$ & $31.24$ & 1995 & R, MVD, CCSD(T) \\
 &  & [\citenum{Goebel1996a}] & $^1S_0, 5d^{10}$ & $33.91 \pm 0.34$ & 1996 & exp. \\
 &  & [\citenum{Seth1997}] & $^1S, 5d^{10}$ & $34.42$ & 1997 & R, PP, CCSD(T) \\
 &  & [\citenum{Roos2005}] & $^1S, 5d^{10}$ & $33.3$ & 2005 & R, DK, CASPT2 \\
 &  & [\citenum{Maroulis2006, Kellö1996}] & $^1S_0, 5d^{10}$ & $34.73 \pm 0.52$ & 2006 & R, DK, CCSD(T) \\
 &  & [\citenum{Pershina2008b}] & $^1S_0, 5d^{10}$ & $34.15$ & 2008 & R, Dirac, CCSD(T) \\
 &  & [\citenum{Kellö1995, Qiao2012, Tang2008}] & $^1S_0, 5d^{10}$ & $33.75$ & 2012 & exp. \\
 &  & [\citenum{Singh2015}] & $^1S_0, 5d^{10}$ & $34.27$ & 2015 & R, Dirac, CCSDT + QED \\
 &  & [\citenum{Borschevsky2015}] & $^1S_0, 5d^{10}$ & $34.1$ & 2015 & R, Dirac, CCSD(T) \\
 &  & [\citenum{Chattopadhyay2015}] & $^1S_0, 5d^{10}$ & $33.59$ & 2015 & R, PRCC(T) \\
 &  & [\citenum{Dyugaev2016}] & $^1S, 5d^{10}$ & $32.9$ & 2016 & semi-empirical \\
 &  & [\citenum{Dzuba2016b}] & $^1S_0, 5d^{10}$ & $39.1$ & 2016 & R, RPA, PolPot \\
 &  & [\citenum{gobre2016efficient}] & $^1S_0, 5d^{10}$ & $33.90$ & 2016 & LR-CCSD \\
 &  & [\citenum{Gould2016a}] & $^1S_0, 5d^{10}$ & $33.5$ & 2016 & TD-DFT (LEXX) \\
 &  & [\citenum{Sahoo2018d}] & $^1S_0, 5d^{10}$ & $34.2 \pm 0.5$ & 2018 & R, CCSD(T) + Breit \\
 &  & [\citenum{Sahoo2018}] & $^1S_0, 5d^{10}$ & $34.5 \pm 0.8$ & 2018 & R, CCSD(T) \\
 &  & [\citenum{A.Manz2019}] & $^1S_0, 5d^{10}$ & $35.45$ & 2019 & ECP, CCSD \\
 &  & [\citenum{Schwerdtfeger2019}] & $--$ & $33.91 \pm 0.34$ & 2019 & recommended \\
 &  & [\citenum{Kumar2021}] & $^1S_0, 5d^{10}$ & $33.69 \pm 0.34$ & 2021 & PRCC(T)+Breitt+QED \\
 &  & [\citenum{Centoducatte2022}] & $^1S_0, 5d^{10}$ & $36.1$ & 2022 & R (ZORA), DFT (B3LYP) \\
 &  & [\citenum{Neto2023}] & $^1S_0, 5d^{10}$ & $34.9$ & 2023 & R (ATZP-ZORA), DFT (B3LYP) \\
81 & Tl & [\citenum{Cernusak2003}] & $^2P$ & $50.48$ & 2003 & R, DK, CCSD(T) \\
 &  & [\citenum{Cernusak2003}] & $^2P$ & $50.62$ & 2003 & R, DK, CCSD(T) \\
 &  & [\citenum{Lide2004}] & $^2P_{1/2}$ & $51.3 \pm 5.4$ & 2004 & exp. \\
 &  & [\citenum{Fleig2005}] & $^2P$ & $70.0$ & 2005 & R, SF, MRCI, $M_L$ res. \\
 &  & [\citenum{Fleig2005}] & $^2P_{1/2}/^2P_{3/2}$ & $51.6/81.2$ & 2005 & R, Dirac, MRCI, $M_J$ res. \\
 &  & [\citenum{Safronova2006}] & $^2P$ & $50.4$ & 2006 & R, DHF, SD, MBPT all-order \\
 &  & [\citenum{Pershina2008a}] & $^2P_{1/2}$ & $52.3$ & 2008 & R, Dirac, FS-CCSD \\
 &  & [\citenum{Dzuba2009}] & $^2P$ & $48.81$ & 2009 & R, Dirac, CI+MBPT \\
 &  & [\citenum{Mitroy2010a, Kozlov2001}] & $^2P$ & $49.2$ & 2010 & RCI + MBPT \\
 &  & [\citenum{Buchachenko2010c}] & $^2P_{1/2}/^2P_{3/2}$ & $50.7 \pm 5.0/78.5 \pm 6.0$ & 2010 & SI-SOCI, $M_J$ res. \\
 &  & [\citenum{Borschevsky2012}] & $^2P_{1/2}/^2P_{3/2}$ & $50.3/80.9$ & 2012 & \makecell{R, Dirac, FSCC, $M_J$ res. \\($J = 3/2$: $M_J = 3/2$: 56.7, \\$M_J = 1/2$: 105.1)} \\
 &  & [\citenum{Hohm2012, Cernusak2003}] & $^2P$ & $71.7 \pm 1.1$ & 2012 & R, DK, CCSD(T) \\
 &  & [\citenum{Borschevsky2012}] & $^2P$ & $52.1 \pm 1.6/80.4 \pm 4.0$ & 2012 & R, Dirac, FSCC \\
 &  & [\citenum{Safronova2013a}] & $^2P$ & $50.0 \pm 1.0$ & 2013 & R, CC \\
 &  & [\citenum{Safronova2013a}] & $^2P$ & $50.7$ & 2013 & R, CI + all-order \\
 &  & [\citenum{Borschevsky2015, Pershina2008a}] & $^2P$ & $51.3$ & 2015 & R, Dirac, FS-CCSD \\
 &  & [\citenum{Dzuba2016}] & $^2P$ & $47.78$ & 2016 & R, Dirac+Breit+QED, SD+CI, RPA \\
 &  & [\citenum{Gould2016a}] & $^2P$ & $51.4$ & 2016 & TD-DFT (LEXX) \\
 &  & [\citenum{gobre2016efficient}] & $^2P$ & $69.92$ & 2016 & LR-CCSD \\
 &  & [\citenum{Tang2018a}] & $^2P$ & $49.2 \pm 2.0$ & 2018 & R, Dirac+Breit, CCSD \\
 &  & [\citenum{A.Manz2019}] & $^2P$ & $70.06$ & 2019 & ECP, CCSD \\
 &  & [\citenum{Schwerdtfeger2019}] & $--$ & $50 \pm 2$ & 2019 & recommended \\
82 & Pb & [\citenum{Doolen1987}] & $^3P$ & $46 \pm 11$ & 1987 & R, Dirac, LDA \\
 &  & [\citenum{Nash2005}] & $^3P_0$ & $51.0$ & 2005 & R, SOPP, CCSD(T) \\
 &  & [\citenum{Thierfelder2008}] & $^3P_0$ & $47.70$ & 2008 & R, Dirac + Gaunt, CCSD(T) \\
 &  & [\citenum{Pershina2008b}] & $^3P_0$ & $46.96$ & 2008 & R, Dirac, CCSD(T) \\
 &  & [\citenum{Thierfelder2008}] & $^3P_0$ & $47.3 \pm 1.9$ & 2008 & R, Dirac + Gaunt, CCSD(T) \\
 &  & [\citenum{Thierfelder2008, Lide2004}] & $^3P_0$ & $47.1 \pm 7.1$ & 2008 & exp. \\
 &  & [\citenum{Borschevsky2015}] & $^3P_0$ & $47.0$ & 2015 & R, Dirac, FS-CCSD \\
 &  & [\citenum{Ma2015}] & $^3P_0$ & $56 \pm 18$ & 2015 & exp. \\
 &  & [\citenum{Dzuba2016}] & $^3P_0$ & $44.04$ & 2016 & R, Dirac + Breit + QED, SD + CI, RPA \\
 &  & [\citenum{Zalialiutdinov2016}] & $^3P_0$ & $46.5$ & 2016 & R, CI + all-order, RPA \\
 &  & [\citenum{Gould2016a}] & $^3P_0$ & $47.9$ & 2016 & TD-DFT (LEXX) \\
 &  & [\citenum{gobre2016efficient}] & $^3P_0$ & $61.80$ & 2016 & LR-CCSD \\
 &  & [\citenum{A.Manz2019}] & $^3P_0$ & $60.07$ & 2019 & EP, CCSD \\
 &  & [\citenum{Schwerdtfeger2019}] & $--$ & $47 \pm 3$ & 2019 & recommended \\
 &  & [\citenum{Flambaum2020}] & $^3P_0$ & $46$ & 2020 & R, CI+MBPT \\
 &  & [\citenum{Oleynichenko2020}] & $^3P_0$ & $47.0$ & 2020 & R, FS-RCCSD(T) \\
 &  & [\citenum{Neto2023}] & $^3P_0$ & $56.3$ & 2023 & R (ATZP-ZORA), DFT(B3LYP) \\
83 & Bi & [\citenum{Kellö1992}] & $^4S$ & $52.85$ & 1992 & R, Cowan-Griffin, HF only \\
 &  & [\citenum{Lide2004, Doolen1987}] & $^4S$ & $50 \pm 12$ & 2004 & R, Dirac, LDA \\
 &  & [\citenum{Roos2004}] & $^4S$ & $48.6$ & 2004 & R, DK, CASPT2 \\
 &  & [\citenum{Buchachenko2010}] & $^4S$ & $48.75$ & 2010 & ECP, CCSD(T) \\
 &  & [\citenum{Ma2015}] & $^4S_{3/2}$ & $55 \pm 11$ & 2015 & exp. \\
 &  & [\citenum{Dzuba2016}] & $^4S$ & $44.62$ & 2016 & R, Dirac + Breit + QED, SD + CI, RPA \\
 &  & [\citenum{Gould2016a}] & $^4S$ & $43.2$ & 2016 & TD-DFT (LEXX) \\
 &  & [\citenum{gobre2016efficient}] & $^4S$ & $49.02$ & 2016 & LR-CCSD \\
 &  & [\citenum{A.Manz2019}] & $^4S$ & $48.88$ & 2019 & ECP, CCSD \\
 &  & [\citenum{Schwerdtfeger2019}] & $--$ & $48 \pm 4$ & 2019 & recommended \\
 &  & [\citenum{Visentin2020}] & $^4S$ & $48.8$ & 2020 & ECP, CCSD(T) \\
 &  & [\citenum{Centoducatte2022}] & $^4S$ & $44.36$ & 2022 & R (ZORA), CCSD(T) \\
 &  & [\citenum{Centoducatte2022}] & $^4S$ & $53.1$ & 2022 & R (ATZP-ZORA), DFT (B3LYP) \\
 &  & [\citenum{Neto2023}] & $^4S$ & $46.6$ & 2023 & R (ZORA), DFT (B3LYP) \\
84 & Po & [\citenum{Kellö1992}] & $^3P_2$ & $46.8$ & 1992 & R, Cowan-Griffin, HF only, $M_L$ res. \\
 &  & [\citenum{Lide2004, Doolen1987}] & $^3P_2$ & $46$ & 2004 & R, Dirac, LDA \\
 &  & [\citenum{Maroulis2006, Hohm2012, Kellö1992}] & $^3P_2$ & $43.6 \pm 4.4$ & 2012 & R, Cowan-Griffin, HF only \\
 &  & [\citenum{gobre2016efficient}] & $^3P_2$ & $45.01$ & 2016 & LR-CCSD \\
 &  & [\citenum{A.Manz2019}] & $^3P_2$ & $44.22$ & 2019 & ECP, CCSD \\
 &  & [\citenum{Schwerdtfeger2019}] & $--$ & $44 \pm 4$ & 2019 & recommended \\
 &  & [\citenum{Centoducatte2022}] & $^3P_2$ & $44.77$ & 2022 & R (ZORA), CCSD(T) \\
 &  & [\citenum{Centoducatte2022}] & $^3P_2$ & $47.1$ & 2022 & R (ZORA), DFT (B3LYP) \\
 &  & [\citenum{Neto2023}] & $^3P_2$ & $46.5$ & 2023 & R (ATZP-ZORA), DFT (B3LYP) \\
85 & At & [\citenum{Fleig2002}] & $^2P_{1/2}$ & $45.6$ & 2002 & R, DK, SO-CI \\
 &  & [\citenum{Fleig2002}] & $^2P_{3/2}$ & $41.9$ & 2002 & R, DK, SO-CI, $M_J$ res. \\
 &  & [\citenum{Maroulis2006, Hohm2012, Kellö1992}] & $^2P_{3/2}$ & $40.7 \pm 2.0$ & 2012 & R, Cowan-Griffin, HF only \\
 &  & [\citenum{gobre2016efficient}] & $^2P_{3/2}$ & $38.93$ & 2016 & LR-CCSD \\
 &  & [\citenum{A.Manz2019}] & $^2P_{3/2}$ & $38.15$ & 2019 & ECP, CCSD \\
 &  & [\citenum{Schwerdtfeger2019}] & $--$ & $42 \pm 4$ & 2019 & recommended \\
 &  & [\citenum{Centoducatte2022}] & $^2P_{3/2}$ & $40.4$ & 2022 & R (ZORA), DFT (B3LYP) \\
 &  & [\citenum{Neto2023}] & $^2P_{3/2}$ & $41.1$ & 2023 & R (ATZP-ZORA), DFT (B3LYP) \\
86 & Rn & [\citenum{Runeberg1998}] & $^1S_0$ & $34.33$ & 1998 & R, SOPP, CCSD(T) + MP2 basis set correction \\
 &  & [\citenum{Runeberg1998}] & $^1S_0$ & $34.60$ & 1998 & R, SOPP, CCSD(T) + MP2 basis set correction \\
 &  & [\citenum{Nakajima2001}] & $^1S$ & $33.18$ & 2001 & R, DK3, CCSD(T) \\
 &  & [\citenum{Soldán2001a}] & $^1S_0$ & $34.4374 \pm 0.0001$ & 2001 & CCSD(T), ECP \\
 &  & [\citenum{Roos2004}] & $^1S$ & $32.6$ & 2004 & R, DK, CASPT2 \\
 &  & [\citenum{Lide2004, Doolen1987}] & $^1S_0$ & $36 \pm 5$ & 2004 & R, Dirac, LDA \\
 &  & [\citenum{Nash2005}] & $^1S_0$ & $28.6$ & 2005 & R, SOPP, CCSD(T) \\
 &  & [\citenum{Wesendrup2000, Labello2005}] & $^1S_0$ & $35.77$ & 2005 & R, DK, CCSD(T) \\
 &  & [\citenum{Labello2005}] & $^1S_0$ & $35.47$ & 2005 & CCSD, ECP \\
 &  & [\citenum{Chattopadhyay2012}] & $^1S_0$ & $35.391$ & 2012 & R, RPA, PolPot \\
 &  & [\citenum{Sulzer2012}] & $^1S_0$ & $35.87$ & 2012 & R, DFT, DC, PBE38 \\
 &  & [\citenum{Pantazis2012}] & $^1S_0$ & $34.89$ & 2012 & R, DKH2, B3LYP, SARC \\
 &  & [\citenum{Pantazis2012}] & $^1S_0$ & $34.70$ & 2012 & R, DKH2, B3LYP, UGBS \\
 &  & [\citenum{Sulzer2012}] & $^1S_0$ & $33.62$ & 2012 & R, DFT, sfDC, PBE38 \\
 &  & [\citenum{Hohm2012, Pershina2008}] & $^1S_0$ & $35.04 \pm 1.8$ & 2012 & R, Dirac, CCSD(T) \\
 &  & [\citenum{Borschevsky2015}] & $^1S_0$ & $35.0$ & 2015 & R, Dirac, CCSD(T) \\
 &  & [\citenum{Dzuba2016b}] & $^1S_0$ & $34.2$ & 2016 & R, RPA, PolPot \\
 &  & [\citenum{Gould2016a}] & $^1S_0$ & $32.2$ & 2016 & TD-DFT (LEXX) \\
 &  & [\citenum{gobre2016efficient}] & $^1S_0$ & $33.54$ & 2016 & LR-CCSD \\
 &  & [\citenum{Sahoo2018, Smits2018}] & $^1S_0$ & $37.0 \pm 0.5$ & 2018 & R, CCSD(T) \\
 &  & [\citenum{Smits2018}] & $^1S_0$ & $35.3$ & 2018 & R, Dirac-Gaunt, CCSD(T) \\
 &  & [\citenum{Dzuba2018a}] & $^1S_0$ & $35.00$ & 2018 & R, RPA \\
 &  & [\citenum{A.Manz2019}] & $^1S_0$ & $32.8$ & 2019 & ECP, CCSD \\
 &  & [\citenum{Schwerdtfeger2019}] & $--$ & $35 \pm 2$ & 2019 & recommended \\
 &  & [\citenum{Flambaum2020}] & $^1S_0$ & $35$ & 2020 & R, RPA \\
 &  & [\citenum{Dutta2020}] & $^1S_0$ & $34.66$ & 2020 & R, DHF, MBPT \\
 &  & [\citenum{McNeill2020}] & $^1S_0$ & $36.14$ & 2020 & R, DK, CCSD(T) \\
 &  & [\citenum{Kumar2021}] & $^1S_0$ & $35.53 \pm 0.36$ & 2021 & PRCC(T)+Breit+QED \\
 &  & [\citenum{Centoducatte2022}] & $^1S_0$ & $34.6$ & 2022 & R (ZORA), DFT (B3LYP) \\
 &  & [\citenum{Neto2023}] & $^1S_0$ & $30.9$ & 2023 & R (ATZP-ZORA), DFT (B3LYP) \\
87 & Fr & [\citenum{Lim2005}] & $^2S$ & $315.2$ & 2005 & R, DK, CCSD(T), AE \\
 &  & [\citenum{Safronova2007}] & $^2S_{1/2}$ & $313.7$ & 2007 & R, DF, RPA, MBPT \\
 &  & [\citenum{Derevianko1999, Derevianko2010}] & $^2S_{1/2}$ & $317.8 \pm 2.4$ & 2010 & R, Dirac, SD all orders + experimental data \\
 &  & [\citenum{Borschevsky2013}] & $^2S_{1/2}$ & $311.5$ & 2013 & R, Dirac, CCSD(T) \\
 &  & [\citenum{gobre2016efficient}] & $^2S_{1/2}$ & $317.8$ & 2016 & LR-CCSD \\
 &  & [\citenum{Singh2016a}] & $^2S_{1/2}$ & $316.8$ & 2016 & R, Dirac-Fock, CCSD(T) \\
 &  & [\citenum{Schwerdtfeger2019}] & $--$ & $317.8 \pm 2.4$ & 2019 & recommended \\
 &  & [\citenum{Aoki2021}] & $^2S_{1/2}$ & $317.1 \pm 1.3$ & 2021 & R, Dirac-HF, CCSD \\
 &  & [\citenum{Smialkowski2021}] & $^2S_{1/2}$ & $325.8$ & 2021 & SR, CCSD(T), ECP \\
 &  & [\citenum{Kaur2022a}] & $^2S_{1/2}$ & $316.6 \pm 2.4$ & 2022 & R, all orders, Dirac-Fock, RPA \\
 &  & [\citenum{Badhan2022}] & $^2S_{1/2}$ & $316.6 \pm 1.5$ & 2022 & R, Dirac-HF, perturbative singles + doubles method, RPA \\
88 & Ra & [\citenum{Lim2004}] & $^1S_0$ & $248.56$ & 2004 & R, DK + SO, CCSD(T) \\
 &  & [\citenum{Maroulis2006, Lim2004}] & $^1S_0$ & $246.2 \pm 4.9$ & 2006 & R, DK + SO, CCSD(T) \\
 &  & [\citenum{Borschevsky2013a}] & $^1S_0$ & $242.8$ & 2013 & R, Dirac + Gaunt, CCSD(T) \\
 &  & [\citenum{Chattopadhyay2014}] & $^1S_0$ & $242.42$ & 2014 & \makecell{R, Dirac + Breit, perturbed relativistic \\coupled-cluster theory (PRCC)} \\
 &  & [\citenum{Dzuba2016b}] & $^1S_0$ & $232$ & 2016 & R, RPA, PolPot \\
 &  & [\citenum{gobre2016efficient}] & $^1S_0$ & $246.2$ & 2016 & LR-CCSD \\
 &  & [\citenum{Sahoo2018}] & $^1S_0$ & $236 \pm 15$ & 2018 & R, CCSD(T) \\
 &  & [\citenum{Bala2019}] & $^1S_0$ & $248.5$ & 2019 & R, KRCISD/aug-QZ \\
 &  & [\citenum{Schwerdtfeger2019}] & $--$ & $246 \pm 4$ & 2019 & recommended \\
 &  & [\citenum{Dutta2020}] & $^1S_0$ & $247.838$ & 2020 & R, DHF, MBPT \\
 &  & [\citenum{Flambaum2020}] & $^1S_0$ & $250$ & 2020 & R, CI+MBPT \\
 &  & [\citenum{Smialkowski2021}] & $^1S_0$ & $250.5$ & 2021 & SR, CCSD(T), ECP \\
89 & Ac & [\citenum{Lide2004, Doolen1987}] & $^2D_{3/2}, 6d^1$ & $217 \pm 44$ & 2004 & R, Dirac, LDA \\
 &  & [\citenum{Dzuba2014}] & $^2D_{3/2}, 6d^1$ & $203.3$ & 2014 & \makecell{R, Dirac, CI + MBPT + CP(RPA); ($\alpha_D$ = 141.9 for the $7s^2 7p^1$ configuration)} \\
 &  & [\citenum{Schwerdtfeger2019}] & $--$ & $203 \pm 12$ & 2019 & recommended \\
 &  & [\citenum{Flambaum2020}] & $^2D_{3/2}, 6d^1$ & $195$ & 2020 & R, CI+MBPT \\
90 & Th & [\citenum{Lide2004, Doolen1987}] & $6d^2$ & $217 \pm 54$ & 2004 & R, Dirac, LDA \\
 &  & [\citenum{Hohm2012}] & $6d^2$ & $166.7$ & 2012 & Estimated from correlation with ionization energies \\
 &  & [\citenum{Schwerdtfeger2019}] & $--$ & $217 \pm 54$ & 2019 & recommended \\
91 & Pa & [\citenum{Lide2004, Doolen1987}] & $5f^2 6d^1$ & $171 \pm 34$ & 2004 & R, Dirac, LDA \\
 &  & [\citenum{Dzuba2014}] & $5f^2 6d^1$ & $154.4$ & 2014 & \makecell{R, Dirac, CI + MBPT + CP(RPA); \\($\alpha_D$ = 151.9 for the $5f^2 6d^2 7s^1$ configuration)} \\
 &  & [\citenum{Schwerdtfeger2019}] & $--$ & $154 \pm 20$ & 2019 & recommended \\
 &  & [\citenum{Flambaum2020}] & $5f^2 6d^1$ & $170$ & 2020 & R, CI+MBPT \\
92 & U & [\citenum{Kadar-Kallen1994}] & $^5L_6, 5f^3 6d^1$ & $137 \pm 9$ & 1994 & exp. \\
 &  & [\citenum{Lide2004, Doolen1987}] & $5f^3 6d^1$ & $153 \pm 38$ & 2004 & R, Dirac, LDA \\
 &  & [\citenum{Dzuba2014}] & $5f^3 6d^1$ & $127.8$ & 2014 & \makecell{R, Dirac, CI + MBPT + CP(RPA); ($\alpha_D$ = 153.2 for the $5f^4$ configuration)} \\
 &  & [\citenum{Schwerdtfeger2019}] & $--$ & $129 \pm 17$ & 2019 & recommended \\
 &  & [\citenum{Flambaum2020}] & $5f^3 6d^1$ & $165$ & 2020 & R, CI+MBPT \\
93 & Np & [\citenum{Lide2004, Doolen1987}] & $5f^4 6d^1$ & $167 \pm 42$ & 2004 & R, Dirac, LDA \\
 &  & [\citenum{Dzuba2014}] & $5f^4 6d^1$ & $150.5$ & 2014 & \makecell{R, Dirac, CI + MBPT + CP(RPA); \\($\alpha_D$ = 127.5 for the $5f^5$ configuration)} \\
 &  & [\citenum{Schwerdtfeger2019}] & $--$ & $151 \pm 20$ & 2019 & recommended \\
 &  & [\citenum{Flambaum2020}] & $5f^4 6d^1$ & $160$ & 2020 & R, CI+MBPT \\
94 & Pu & [\citenum{Lide2004, Doolen1987}] & $5f^6$ & $165 \pm 41$ & 2004 & R, Dirac, LDA \\
 &  & [\citenum{Dzuba2014}] & $5f^6$ & $132.2$ & 2014 & \makecell{R, Dirac, CI + MBPT + CP(RPA); \\($\alpha_D$ = 147.6 for the $5f^5 6d^1$ configuration)} \\
 &  & [\citenum{Schwerdtfeger2019}] & $--$ & $132 \pm 20$ & 2019 & recommended \\
 &  & [\citenum{Flambaum2020}] & $5f^6$ & $144$ & 2020 & R, CI+MBPT \\
95 & Am & [\citenum{Lide2004, Doolen1987}] & $5f^7$ & $157 \pm 39$ & 2004 & R, Dirac, LDA \\
 &  & [\citenum{Roos2005a}] & $5f^7$ & $116 \pm 29$ & 2005 & R, DK, CASPT2 \\
 &  & [\citenum{Dzuba2014}] & $5f^7$ & $131.2$ & 2014 & \makecell{R, Dirac, CI + MBPT + CP(RPA); \\($\alpha_D$ = 144.7 for the $5f^6 6d^1$ configuration)} \\
 &  & [\citenum{Martins2016}] & $5f^7$ & $122.4$ & 2016 & R, DFT, DKH, B3LYP \\
 &  & [\citenum{Schwerdtfeger2019}] & $--$ & $131 \pm 25$ & 2019 & recommended \\
96 & Cm & [\citenum{Lide2004, Doolen1987}] & $5f^7 6d^1$ & $155 \pm 39$ & 2004 & R, Dirac, LDA \\
 &  & [\citenum{Dzuba2014}] & $5f^7 6d^1$ & $143.6$ & 2014 & \makecell{R, Dirac, CI + MBPT + CP(RPA); \\($\alpha_D$ = 128.6 for the $5f^8$ configuration)} \\
 &  & [\citenum{Schwerdtfeger2019}] & $--$ & $144 \pm 25$ & 2019 & recommended \\
97 & Bk & [\citenum{Lide2004, Doolen1987}] & $5f^9$ & $153 \pm 38$ & 2004 & R, Dirac, LDA \\
 &  & [\citenum{Dzuba2014}] & $5f^9$ & $125.3$ & 2014 & \makecell{R, Dirac, CI + MBPT + CP(RPA); \\($\alpha_D$ = 141.6 for the $5f^8 6d^1$ configuration)} \\
 &  & [\citenum{Schwerdtfeger2019}] & $--$ & $125 \pm 25$ & 2019 & recommended \\
98 & Cf & [\citenum{Lide2004, Doolen1987}] & $5f^{10}$ & $138 \pm 34$ & 2004 & R, Dirac, LDA \\
 &  & [\citenum{Dzuba2014}] & $5f^{10}$ & $121.5$ & 2014 & \makecell{R, Dirac, CI + MBPT + CP(RPA); \\($\alpha_D$ = 142.3 for the $5f^9 6d^1$ configuration)} \\
 &  & [\citenum{Schwerdtfeger2019}] & $--$ & $122 \pm 20$ & 2019 & recommended \\
99 & Es & [\citenum{Lide2004, Doolen1987}] & $5f^{11}$ & $133 \pm 33$ & 2004 & R, Dirac, LDA \\
 &  & [\citenum{Dzuba2014}] & $5f^{11}$ & $117.5$ & 2014 & \makecell{R, Dirac, CI + MBPT + CP(RPA); \\($\alpha_D$ = 146.1 for the $5f^{{10}} 6d^1$ configuration)} \\
 &  & [\citenum{Schwerdtfeger2019}] & $--$ & $118 \pm 20$ & 2019 & recommended \\
100 & Fm & [\citenum{Lide2004, Doolen1987}] & $5f^{12}$ & $161 \pm 40$ & 2004 & R, Dirac, LDA \\
 &  & [\citenum{Dzuba2014}] & $5f^{12}$ & $113.4$ & 2014 & \makecell{R, Dirac, CI + MBPT + CP(RPA); \\($\alpha_D$ = 155.6 for the $5f^{{11}} 6d^1$ configuration)} \\
 &  & [\citenum{Schwerdtfeger2019}] & $--$ & $113 \pm 20$ & 2019 & recommended \\
101 & Md & [\citenum{Lide2004, Doolen1987}] & $5f^{13}$ & $123 \pm 31$ & 2004 & R, Dirac, LDA \\
 &  & [\citenum{Dzuba2014}] & $5f^{13}$ & $109.4$ & 2014 & \makecell{R, Dirac, CI + MBPT + CP(RPA); \\($\alpha_D$ = 179.6 for the $5f^{{12}} 6d^1$ configuration)} \\
 &  & [\citenum{Schwerdtfeger2019}] & $--$ & $109 \pm 20$ & 2019 & recommended \\
102 & No & [\citenum{Lide2004, Doolen1987}] & $^1S_0, 5f^{14}$ & $118 \pm 30$ & 2004 & R, Dirac, LDA \\
 &  & [\citenum{Thierfelder2009}] & $^1S_0, 5f^{14}$ & $110.8 \pm 5.5$ & 2009 & R, Dirac + Gaunt, CCSD(T) \\
 &  & [\citenum{Thierfelder2009}] & $^1S_0, 5f^{14}$ & $115.64$ & 2009 & R, DK, CCSD(T) \\
 &  & [\citenum{Dzuba2014}] & $^1S_0, 5f^{14}$ & $105.4$ & 2014 & \makecell{R, Dirac, CI + MBPT + CP(RPA); \\($\alpha_D$ = 267.8 for the $5f^{14} 7s 7p^1$ configuration)} \\
 &  & [\citenum{Dzuba2014, Dzuba2014a}] & $^1S_0, 5f^{14}$ & $112 \pm 6$ & 2014 & R, DF, CI + MBPT + Breit + QED \\
 &  & [\citenum{Dzuba2014, Dzuba2014a}] & $^1S_0, 5f^{14}$ & $110 \pm 8$ & 2014 & R, DF, CI + all-order + Breit + QED \\
 &  & [\citenum{Dzuba2016b}] & $^1S_0, 5f^{14}$ & $114$ & 2016 & R, RPA, PolPot \\
 &  & [\citenum{Yoshizawa2016}] & $^1S_0, 5f^{14}$ & $107.77$ & 2016 & R, DFT, CAM-B3LYP, 2c-NESC \\
 &  & [\citenum{Martins2016}] & $^1S_0, 5f^{14}$ & $115.6$ & 2016 & R, DFT, DKH, B3LYP \\
 &  & [\citenum{Schwerdtfeger2019}] & $--$ & $110 \pm 6$ & 2019 & recommended \\
103 & Lr & [\citenum{Dzuba2014a}] & $7p^1$ & $323 \pm 80$ & 2014 & R, DF, CI + all-order + Breit + QED \\
 &  & [\citenum{Dzuba2014a}] & $7p^1$ & $320 \pm 80$ & 2014 & R, DF, CI + MBPT + Breit + QED \\
 &  & [\citenum{Srivastava2016}] & $7p^1$ & $225.2$ & 2016 & R, DK, DFT, CAM-B3LYP \\
 &  & [\citenum{Schwerdtfeger2019}] & $--$ & $320 \pm 20$ & 2019 & recommended \\
104 & Rf & [\citenum{Dzuba2014a}] & $6d^2$ & $107 \pm 5$ & 2014 & R, DF, CI + MBPT + Breit + QED \\
 &  & [\citenum{Dzuba2014a}] & $6d^2$ & $115 \pm 13$ & 2014 & R, DF, CI + all-order + Breit + QED \\
 &  & [\citenum{Schwerdtfeger2019}] & $--$ & $112 \pm 10$ & 2019 & recommended \\
105 & Db & [\citenum{Dzuba2016b}] & $6d^3$ & $42.5$ & 2016 & R, RPA, PolPot \\
 &  & [\citenum{Dzuba2016b}] & $6d^3$ & $42 \pm 4$ & 2016 & R, RPA, PolPot (value recommended by authors) \\
 &  & [\citenum{Schwerdtfeger2019}] & $--$ & $42 \pm 4$ & 2019 & recommended \\
106 & Sg & [\citenum{Dzuba2016b}] & $6d^4$ & $40.7$ & 2016 & R, RPA, PolPot \\
 &  & [\citenum{Dzuba2016b}] & $6d^4$ & $40 \pm 4$ & 2016 & R, RPA, PolPot (value recommended by authors) \\
 &  & [\citenum{Schwerdtfeger2019}] & $--$ & $40 \pm 4$ & 2019 & recommended \\
107 & Bh & [\citenum{Dzuba2016b}] & $6d^5$ & $38.4$ & 2016 & R, RPA, PolPot \\
 &  & [\citenum{Dzuba2016b}] & $6d^5$ & $38 \pm 4$ & 2016 & R, RPA, PolPot (value recommended by authors) \\
 &  & [\citenum{Schwerdtfeger2019}] & $--$ & $38 \pm 4$ & 2019 & recommended \\
108 & Hs & [\citenum{Dzuba2016b}] & $6d^6$ & $36.2$ & 2016 & R, RPA, PolPot \\
 &  & [\citenum{Dzuba2016b}] & $6d^6$ & $36 \pm 4$ & 2016 & R, RPA, PolPot (value recommended by authors) \\
 &  & [\citenum{Schwerdtfeger2019}] & $--$ & $36 \pm 4$ & 2019 & recommended \\
109 & Mt & [\citenum{Dzuba2016b}] & $6d^7$ & $34.2$ & 2016 & R, RPA, PolPot \\
 &  & [\citenum{Dzuba2016b}] & $6d^7$ & $34 \pm 3$ & 2016 & R, RPA, PolPot (value recommended by authors) \\
 &  & [\citenum{Schwerdtfeger2019}] & $--$ & $34 \pm 3$ & 2019 & recommended \\
110 & Ds & [\citenum{Dzuba2016b}] & $6d^8$ & $32.3$ & 2016 & R, RPA, PolPot \\
 &  & [\citenum{Dzuba2016b}] & $6d^8$ & $32 \pm 3$ & 2016 & R, RPA, PolPot (recommended value by authors) \\
 &  & [\citenum{Schwerdtfeger2019}] & $--$ & $32 \pm 3$ & 2019 & recommended \\
111 & Rg & [\citenum{Seth1996}] & $6d^9$ & $31.6$ & 1996 & ARPP, CCSD(T) \\
 &  & [\citenum{Dzuba2016b}] & $6d^9$ & $30.6$ & 2016 & R, RPA, PolPot \\
 &  & [\citenum{Dzuba2016b}] & $6d^9$ & $30 \pm 3$ & 2016 & R, RPA, PolPot (value recommended by authors) \\
 &  & [\citenum{Schwerdtfeger2019}] & $--$ & $32 \pm 6$ & 2019 & recommended \\
112 & Cn & [\citenum{Seth1997}] & $^1S_0, 6d^{10}$ & $25.82$ & 1997 & R, PP, CCSD(T) \\
 &  & [\citenum{Nash2005}] & $^1S_0, 6d^{10}$ & $28.68$ & 2005 & R, SOPP, CCSD(T) \\
 &  & [\citenum{Pershina2008b}] & $^1S_0, 6d^{10}$ & $27.64$ & 2008 & R, Dirac, CCSD(T) \\
 &  & [\citenum{Pershina2008b}] & $^1S_0, 6d^{10}$ & $27.40$ & 2008 & R, Dirac, CCSD(T) \\
 &  & [\citenum{Dzuba2016b}] & $^1S_0, 6d^{10}$ & $28.2$ & 2016 & R, RPA, PolPot \\
 &  & [\citenum{Dzuba2016b}] & $^1S_0, 6d^{10}$ & $28 \pm 4$ & 2016 & R, RPA, PolPot (value recommended by authors) \\
 &  & [\citenum{Schwerdtfeger2019}] & $--$ & $28 \pm 2$ & 2019 & recommended \\
 &  & [\citenum{Kumar2021}] & $^1S_0, 6d^{10}$ & $27.44 \pm 0.88$ & 2021 & PRCC(T)+Breit+QED \\
113 & Nh & [\citenum{Pershina2008a}] & $^2P_{1/2}$ & $29.85$ & 2008 & R, Dirac, FS-CCSD \\
 &  & [\citenum{Dzuba2016}] & $^2P_{1/2}$ & $28.8$ & 2016 & R, Dirac+Breit+QED, SD+CI, RPA \\
 &  & [\citenum{Schwerdtfeger2019}] & $--$ & $29 \pm 2$ & 2019 & recommended \\
114 & Fl & [\citenum{Nash2005}] & $^3P_0$ & $34.35$ & 2005 & R, SOPP, CCSD(T) \\
 &  & [\citenum{Thierfelder2008}] & $^3P_0$ & $31.87$ & 2008 & R, Dirac + Gaunt, CCSD(T) \\
 &  & [\citenum{Pershina2008b}] & $^3P_0$ & $30.59$ & 2008 & R, Dirac, CCSD(T) \\
 &  & [\citenum{Pershina2008b}] & $^3P_0$ & $29.52$ & 2008 & estimate \\
 &  & [\citenum{Thierfelder2008}] & $^3P_0$ & $31.0$ & 2008 & R, Dirac + Gaunt, CCSD(T) \\
 &  & [\citenum{Dzuba2016}] & $^3P_0$ & $31.4$ & 2016 & R, Dirac + Breit + QED, SD + CI, RPA \\
 &  & [\citenum{Schwerdtfeger2019}] & $--$ & $31 \pm 4$ & 2019 & recommended \\
115 & Mc & [\citenum{Pershina2014}] & $^4S_{3/2}$ & $66$ & 2014 & Estimated via correlation with $R_\text{max}(np_{3/2})$ \\
 &  & [\citenum{Dzuba2016}] & $^4S_{3/2}$ & $70.5$ & 2016 & R, Dirac + Breit + QED, SD + CI, RPA \\
 &  & [\citenum{Schwerdtfeger2019}] & $--$ & $71 \pm 20$ & 2019 & recommended \\
116 & Lv & [\citenum{Pershina2014}] & $^3P_2$ & $61.17$ & 2014 & Estimated via correlation with $R_\text{max}(np_{3/2})$ \\
 &  & [\citenum{Schwerdtfeger2019}] & $--$ & $67 \pm 10$ & 2019 & recommended \\
117 & Ts & [\citenum{Pershina2014}] & $^2P_{3/2}$ & $52.24$ & 2014 & Estimated via correlation with $R_\text{max}(np_{3/2})$ \\
 &  & [\citenum{deFarias2017}] & $^2P_{3/2}$ & $76.3$ & 2017 & empirical estimate \\
 &  & [\citenum{Schwerdtfeger2019}] & $--$ & $76 \pm 15$ & 2019 & recommended \\
118 & Og & [\citenum{Nash2005}] & $^1S_0$ & $52.4$ & 2005 & R, SOPP, CCSD(T) \\
 &  & [\citenum{Pershina2008}] & $^1S_0$ & $46.33$ & 2008 & R, Dirac, CCSD(T) \\
 &  & [\citenum{Dzuba2016b}] & $^1S_0$ & $59.0/57.2$ & 2016 & R, RPA, PolPot \\
 &  & [\citenum{Dzuba2016b}] & $^1S_0$ & $57 \pm 3$ & 2016 & R, RPA, PolPot \\
 &  & [\citenum{Jerabek2018}] & $^1S_0$ & $57.98$ & 2018 & R, Dirac + Gaunt, CCSD(T) \\
 &  & [\citenum{Schwerdtfeger2019}] & $--$ & $58 \pm 6$ & 2019 & recommended \\
 &  & [\citenum{Kumar2021}] & $^1S_0$ & $56.5 \pm 1.8$ & 2021 & PRCC(T)+Breit+QED \\
119 & Uue & [\citenum{Lim1999}] & $^2S$ & $169$ & 1999 & R, Dirac, CCSD(T) \\
 &  & [\citenum{Lim2005}] & $^2S$ & $163.7$ & 2005 & R, DK, CCSD(T), ARPP \\
 &  & [\citenum{Lim2005}] & $^2S$ & $166.0$ & 2005 & R, DK, CCSD(T), AE \\
 &  & [\citenum{Borschevsky2013}] & $^2S_{1/2}$ & $169.7$ & 2013 & R, Dirac, CCSD(T) \\
 &  & [\citenum{Schwerdtfeger2019}] & $--$ & $169 \pm 4$ & 2019 & recommended \\
120 & Ubn & [\citenum{Borschevsky2013a}] & $^1S_0$ & $162.6$ & 2013 & R, Dirac + Gaunt, CCSD(T) \\
 &  & [\citenum{Dzuba2016b}] & $^1S_0$ & $147$ & 2016 & R, RPA, PolPot \\
 &  & [\citenum{Dzuba2016b}] & $^1S_0$ & $159 \pm 10$ & 2016 & R, RPA, PolPot \\
 &  & [\citenum{Schwerdtfeger2019}] & $--$ & $159 \pm 10$ & 2019 & recommended \\
\end{longtable}


    \bibliography{references}

\end{document}
